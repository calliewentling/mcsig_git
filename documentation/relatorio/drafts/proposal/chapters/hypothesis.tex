The goal of this project is to explore if there is a value to plotting the spatial distribution of news story contents, as well as provide users a portal through which to interact with the data to extract the desired information.  

It is expected that the association of specific place (potentially non-conforming to existing administrative boundaries or defined points of interest) to traditional news articles will provide an added dimension of understanding to communities at a local level. Users will find additional insights from the ability to view or filter spatial attributes, especially in conjunction with thematic and temporal attributes. This type of data preparation, though it is initially cumbersome to establish and requires adjustment of publishers' processes to maintain, will provide a powerful foundation from which future economic (improved publisher products elevating their offering and attracting/maintaining a customer base), societal (illumination of local trends requiring intervention, improved community engagement of readers with their surroundings, or improved city resources), and academic (improved research functionality) benefits may stem.  If this type of functionality and improved user experience are well-implemented by a handful of productive news services, it will force a shift of the industry standard towards integration of spatial attributes and spatially related products.

Though testing of these hypotheses through rigerous comparison to the status quo (traditional online news sources without a spatial element) and emerging product performing automatic extraction of place (such those of the GDELT project, Section \ref{appendix:existing_projects}) are not included in this endeavor, the resulting tools should provide a basis from which future projects may develop and evaluate.