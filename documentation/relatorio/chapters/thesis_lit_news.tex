\subsection{News localization}
%- {\color{purple}News is a distribution of information to humans which in turn affects their actions. We have manners of measuring the physical world via sensors, and now are layering in emotional responses. The news is a feedback loop that is at once an input to human decision making and a report of some of those outputs.} \\
%- {\color{orange}“Everything happens somewhere and some-whence. Spatial and temporal aspects of data lead to critical insights into the information contained in it.”}\cite{Bhattacharya2018}\\
%- {\color{orange} “in the current hypermodern context, a smart city must also be able to identify the main components of an event (where, when, who, what, how), to analyze it, to provide location-aware (and contextual) sense to it and, to react (actuate) properly (and in real-time - at least compliant with the nature of this event).} \cite{Roche2012}\\
%-{\color{orange}“Spatio-temporal aspects of data lead to critical information.”}\cite{Bhattacharya2018}\\
%-{\color{orange}“A media or news desert is an uncovered geographical area that has few or no news outlets and receives little coverage. Mapping locations mentioned in news articles is the primary step in identifying news deserts.”\cite{Gupta2020}}\\
%
\subsubsection{News evolution}
%-{\color{orange} “There’s a lot of opportunity for innovation across the news ecosystem, and this is also true for local publishers… They’re connecting the dots with their communities and are trying to build innovative and more engaging experiences using local data.” LB\cite{DNIFund2018}}\\
%-{\color{orange}“Having the tools to access and analyse this data is now critical for effective reporting… There are many powerful public interest stories out there that will only be discovered if traditional investigative techniques are combined with technology” Megan Lucero \cite{DNIFund2018}}\\
%-{\color{orange}Linked in provided a business skillset “which you don’t often get in journalism because we’re, you know, so adamant about separating the editorial and commercial side of things.” Now this is more present in journalism (as of the past 5 years).\cite{Granger2020a}}\\
%-{\color{orange}“I really believe in the community element of journalism. So I’m really interested in not just broadcasting but hearing back and bringing people together. It not where I want it to be, but I think by calling it a community I’m stating that intention.”\cite{Granger2020a}}\\
%-{\color{orange} Matching product to audience: “You have to throw a lot of lines out in the water and see which ones catch. So you end up having a lot of platforms.”\cite{Granger2020a}}\\
%
\subsubsection{Personalization}
%-{\color{orange}“While personalization of content is recognized as a powerful advantage for publishers and readers alike, it can mean that people aren’t often exposed to a range of viewpoints on a particular subject.''\cite{DNIFund2018}}\\
%
\subsubsection{Blogs}
%-{\color{orange}“Blogs, as social software, are designed for rapid content creation, archiving, and syndication within online communities. Due to the community intelligence, blogs have lately become a valuable source of information. Blogs, used as collaboration environments, support people in associating tags with content that they generate, share, or consume within a community. More generally, by tagging content in the web, one makes this content shareable to other web participants, as well as links it to other web contents.” }\cite{Xing2015}\\
%-{\color{orange}“There has been very little research on geoparsing microblog text.”\cite{Karimzadeh2019a}}\\
%
\subsubsection{Indie journalism}
%-{\color{orange} Indie journalism: ”independent side-projects” as entrepreneurial endeavors of journalists.\cite{Granger2020a}}\\
%
%
\subsubsection{Local journalism}
%-{\color{orange} “The news industry may be in flux, but it’s vitally important that good local journalism doesn’t disappear.” Megan Lucero, Director at The Bureau Local\cite{DNIFund2018}}\\
%-{\color{orange}“Local journalists play an important role in democratic society by holding power to account. However, newsrooms have fewer resources to dedicate to much-needed investigative reporting.”\cite{DNIFund2018}}\\
%-{\color{orange}“While almost every reader survey concludes that local news is the most highly valued of all content, making it pay is another matter entirely.”\cite{DNIFund2018}}\\
%-{\color{orange}“The appetite for hyperlocal news is vast”\cite{DNIFund2018}}\\
%-{\color{orange} “The core problem in local and hyperlocal news is that there is no sustainable business model,” says Gabriel Kahn. “These areas have no voice, they have no way of discussing problems in the community and, whether it’s governments or companies, they have no accountability.”\cite{Granger2020}}\\
%-{\color{orange}“Local news is a medium for communication among residents, and they reflect the nature of local geographical knowledge. THe residents’ local knowledge concerning their living environment is often invisible, descriptive and vague, and thus difficult to collect.”\cite{Cai2016}}\\
%-{\color{orange}“Local newspapers can be considered as one important tool to support the accumulation and sharing of local knowledge.”\cite{Cai2016}}\\
%-{\color{orange}“Because local news serves the important role of being repositories of political, social, and cultural knowledge in the community of practice, it is important to make such knowledge accessible to community members who can maintain the awareness of what the community have known or done in the past when facing a new or repeating problem.”\cite{Cai2016}}\\
%-{\color{orange}“citizens’ attention to local news promoted political participation.”\cite{Cai2016}}\\
%-{\color{orange}“Local news are textual artifacts of human experience on the places they live and interact with, and they play an important role in the making of of place meaning. They reflect what local people think and are written for the local people”\cite{Cai2016}}\\
%-{\color{orange} “Hyperlocality is a concept that has emerged in both online journalism and in mobile advertising circles to designate community-oriented digital content endemic to small geographies such as neighbourhoods, streets, or single postcodes.”\cite{Leszczynski2019}}\\
%
%
\subsubsection{Georeferenced News}
%-{\color{orange}“Once a blog system organizes the entries with geographical semantics and display localities, spatial analysis can be performed based on the user-tagged information. In order to develop such system, the domain specific ontologies that take geographical knowledge into account should be built.”}\cite{Xing2015}\\
%-{\color{orange}“In the era of web 2.0, blog systems such as GeoBlog provide an approach for individuals to input variou geographical data and the asynchronous functions to meet the asynchronous interaction requirements. This apparently widens the data source of geographical information applications.}\cite{Xing2015}\\
%-{\color{orange}“Considering the map as an object that conveys meaning, we had to recognise not only the agency of the participants but the agency of the object itself. Yes, this mapping of experiences has the potential to be a powerful source of information, and it also has the power to label places as safe, dangerous, racist, oppressive, accepting, and so forth.” \cite{McQueenBaker2019}}\\
%-{\color{orange}“Participants and researchers come with assumptions wrapped in powerful collective memories. Acknowledging them is important, but we also need to look for disturbances and consider the impact of those disturbances.”\cite{McQueenBaker2019}}\\
%-{\color{orange}“If you are a journalist for a news outlet where location is a key part of the story (such as a local or hyperlocal news site), it is worth geotagging your stories to help search engines deliver local results:”\cite{Marshall2012}}\\
%-{\color{orange}“Location-based feeds have huge potential. They may further develop as a way Google and other search engines return results based on a person’s location; they may be used by social newsreader apps as a way of delivering \cite{Marshall2012}}\\
%-{\color{orange}“Political news reports are populated all over the world in various languages It has a great value to automatically detect the geolocation from these reports for better understanding of the associated events.”\cite{Imani2019}}\\
%-{\color{orange}“Primary focus location as the actual location where the event occurred amongst other focus locations mentioned in the report.”\cite{Imani2019}}\\
%-{\color{orange}“determining crime pattern locations, predicting the place of protests and political unrest, and identifying the geolocation of natural disasters. Such applications can largely benefit form identifying precise geolocation information in a timely manner to provide better support for decision making.”\cite{Imani2019}}\\
%-{\color{orange}“News articles contain a wealth of implicit geographic content that if exposed to readers improves understanding of today’s news.”\cite{Teitler2008}}\\
%-{\color{orange}“Given that much of the interest in news is motivated by location-related attributes of readers (e.g. where readers are situated, ahil from, aspire to be), it is somewhat surprising that they cannot deal easily with the two most common types of spatially-related queries: 1. Feature-based - ‘Where did story X happen?’ 2. Location-based - ‘What is happening in Location Y?’”\cite{Teitler2008}}\\
%-{\color{orange}The traditional newspaper layout “is linear and static, whereas the ap interface is dynamic, in that the articles associated with a particular location can vary over time without disturbing the positioning of other articles.”\cite{Teitler2008}}\\
%-{\color{orange}“an audience’s local lexicon plays a key role in how news authors write for their audiences.” \cite{Lieberman2010}}\\
%-{\color{orange}“geo-referencing local news can contribute to improved accessibility of community knowledge.”\cite{Cai2016}}\\
%-{\color{orange}“For newspapers targeted at local communities, geographical locations are likely to be in finer granularity and demand gazetteer-based approach” (vs. statistical language model approach)\cite{Cai2016}}\\
%-{\color{orange}“Due to uniqueness of each locality, there is a lack of local gazetteers that reflect the richness and details of local spatial language.”\cite{Cai2016}}\\
%-{\color{orange}“Location ambiguity in the local has its own unique nature that presents both challenges and opportunities to resolve them.”\cite{Cai2016}}\\
%-{\color{orange}In this study, only 16 percent of references could be mapped without ambiguity, 59\% were discernible with the applied heuristics, and the remaining 12\% “rely on human annotation to create footprints.”\cite{Cai2016}}\\
%-{\color{orange} Therefore, the gazetteer used for geo-referencing local newspaper articles should be place-specific.”\cite{Cai2016}}\\
%-{\color{orange}“What kinds of spatial language are used in the locality? Answers to this question will make clear the nature of gazetteers needed for geoparsing local newspaper.”\cite{Cai2016}}\\
%-{\color{orange} “Local news articles are an important source of knowledge about local events, place-specific culture, and peoples’ thoughts about their environment. Reliable geocoding of such articles is the first step towards unlocking such local knowledge for community engagement and development.”\cite{Cai2016}}\\
%-{\color{orange} “existing geo-referencing methods and tools do not work well for local news because they do not reflect the ways local people encode and communicate geographical knowledge.”\cite{Cai2016}}\\
%-{\color{orange}“FOr human coders, locating events by reading a news article may be time-intensive, but straightforward. This is not the case for machine-coding: many news articles contain multiple location names, such as the location of the journalist writing the story, the birthplace of a  person being interviewed, or the place of a similar event that occurred several decades ago; at times, human names are identical to geographic names,; and location names are transliterated into English in a variety of potentially confusing ways. All these sources of noise in the data increase the difficulty in automatically locating events.”\cite{Lee2019}}\\
%-{\color{orange}“Further complicating matters, new articles often use nearby landmarks to indicate the location, in lieu of using the official names.”\cite{Lee2019}}\\
%
\subsubsection{Automated location extraction from published corpora}
%-{\color{orange}“The networks are global and spatial proximity is no longer a determinant of relations established in communities, as it used to be. These new communication models allow us to be in the world, however we run the risk of not reaching our street, or our city, and in it to see the reflection of the image of what we desire our urban environment to be.”} \cite{Painho2013}\\
%-{\color{orange}“To map local news coverage, it is important to extract precise location mentions from textual news content.”\cite{Gupta2020}}\\
%-{\color{orange}The traditional newspaper layout “forces readers to perform a brute force sequential search (i.e. read the various articles while looking for mentions of the locations which interest them).”\cite{Teitler2008}}\\
%-{\color{orange}“Determining the geographic focus of a document can be challenging, as not all documents have an easily identifiable focus, and not all locations referenced in a document may be related to its focus. For example, news articles often contain the address of the newspaper that published the articles.”\cite{Teitler2008}}\\
%-{\color{orange}“One related issue is that stories may continually change and be updated, even after they have been ‘published’ in an RSS feed.”\cite{Teitler2008}}\\
%-{\color{orange}“As retrieved, the story webpage is unsuitable for article processing, as it was meant to be read by humans and contains extraneous formatting markup and rendering scripts…. Furthermore, the extraction must be independent of the source website, as it is infeasible to create custom extraction rules for each individual website.”\cite{Teitler2008}}\\
%-{\color{orange} “The successful execution of location-based and feature-based queries on spatial databases requires the construction of spatial indexes on the spatial attributes. This is not simple when the data is unstructured as is the case when the data is a collection of documents such as news articles, which is the domain of discourse, where the spatial attribute consists of text that can be (but is not required to be) interpreted as the names of locations.
%\cite{Lieberman2010}}\\
%-{\color{orange}“news articles (and more generally, documents on the Internet) are written to be understood by a human audience, and therefore geotagging will benefit from processing (i.e., reading) the document in the same way as an intended reader).”\cite{Lieberman2010}}\\
%-{\color{orange}“the reader’s spatial lexicon - those locations that the reader can identify and place on the map without any evidence - is very limited. In fact, even more importantly, this inherent limitation means that a common spatial lexicon shared by all humans cannot exist, which is one of the key principles used by systems such as MetaCarta and Web-a-Where.”\cite{Lieberman2010}}\\
%-{\color{orange}“the existence of a reader’s local spatial lexicon or simply local lexicon that differs from place to place, and that it is separate from a global lexicon of prominent places known by everyone.”\cite{Lieberman2010}}\\
%-{\color{orange}“The local lexicon is even more necessary when geographically indexing locations with smaller spatial extent which correspond to address intersections…, since street names are even more ambiguous than regular toponyms.”\cite{Lieberman2010}}\\
%-{\color{orange}“While statistical NER methods can be useful for analysis of static corpora, they are not well-suited to the dynamic and ever-changing nature of the news”\cite{Lieberman2010}}\\
%-{\color{orange}“The main idea behind our geotagging framework within an article, to make it easier for human readers in the author’s intended audience to recognize and resolve toponyms. Authors create this framework by using linguistic contextual clues that we can detect using heuristic rules. Furthermore, readers are expected to read articles linearly, so article language has a contextual and geographic flow. Toponyms mentioned in a sentence will establish a geographic framework for subsequent text.”\cite{Lieberman2010}}\\
%-{\color{orange}“Certain phrases in article text denote relative geography, which is language that defines a usually imprecise geographic region in terms of distance from or proximity to another geographic location. These imprecise regions are important because they usually target the geographic areas where the events in an article took place, and therefore are useful for resolving the article’s toponyms… We refer to the tononyms in such phrases as anchor toponyms, and we term the resulting regions as target regions.”\cite{Lieberman2010}}\\
%%-{\color{orange}“precision measures how many reported toponyms are correct, but says nothing of how many went unreported. In contrast, recall measures how many ground truth toponyms were reported and correct, but does not indicate how many of all reported toponyms are correct.” F1 score combines these (“harmonic mean of precision and recall”)\cite{Lieberman2010}}\\
%-{\color{orange}Recall drops “reflects the fact that most gazetteers are still rather incomplete or at least not in sync with the frequency of use of location descriptions that do not have formally defined boundaries, such as ‘New England’ and ‘Upper West Side’.”\cite{Lieberman2010}}\\
%-{\color{orange}“Event extraction from news articles is a commonly required prerequisite for various tasks, such as article summarization, article clustering, and news aggregation. Due to the lack of universally applicable and publicly available methods tailored to news datasets, many researchers redundantly implement event extraction methods for their own projects.”\cite{Hamborg2019}}\\
%-{\color{orange}“The extraction of a news article’s main event is an automated analysis task at the core of a range of use cases, including news aggregation, clustering of articles reporting on the same event, and news summarization.”\cite{Hamborg2019}}\\
%-{\color{orange}“Beyond computer science, other disciplines also analyze how news outlets report on events in what is known as frame analyses.”\cite{Hamborg2019}}\\
%-{\color{orange}“Explicit event descriptors are properties that occur in a text to describe an event, e.g., the phrases in an article that enable a reader to understand what the article is reporting on.”\cite{Hamborg2019}}\\
%-{\color{orange}“State-of-the-art  methods for extracting events from articles suffer from three main shortcomings. First, most approaches only detect events implicitly, eg.g. By  employing topic modeling. Second, they are specialized for the extraction of task-specific properties, e.g., extracting only the number of injured people in an attack. Lastly, some methods extract explicit descriptors, but are not publicly available, or are described in insufficient detail to allow researchers to reimplement the approaches.” \cite{Hamborg2019}}\\
%-{\color{orange}“Similar to temporal phrases, locality phrases are often heterogeneous, i.e., they do not only contain temporal NEs but also function words.” Nomanatim: geocode leveraging OSM\cite{Hamborg2019}}\\
%-{\color{orange}“For ‘when’ and ‘where’ questions, we found that in some cases an article does not explicitly mention the main event’s date or location. The date of an event may be implicitly defined by the reported event, e.g., ‘in the final of the Canberra Classic.’. The location may be implicitly defined by the main actor, e.g., ‘Apple Postpones Release of [...]’, which likely happened at the Apple Headquarters in Cupertino. Similarly, the proper noun ‘Stanford University’ also defines a location.”\cite{Hamborg2019}}\\
%-{\color{orange} “The journalistic 5W1H questions are capable of describing the main event of an article, i.e., by answering who did what, when, where, why, and how.”\cite{Hamborg2019}}\\
%-{\color{orange}“Answering the 5W1H questions is at the core of understanding any article, and thus an essential task in many research efforts that analyze articles.”\cite{Hamborg2019}}\\
%-{\color{orange}Class imbalance: “often degrades machine learning approaches, which skews the classification capacity in favor of the most crowded classes.” (applicatin to tags)\cite{Rivera2020}}\\
%-{\color{orange}Most previous studies are english based\cite{Rivera2020}} {\color{purple}innapropriate for local investigations in other langauges}\\
%
%\subsubsection{Examples}
%GLOBAL
%
%{GDELT} {\color{red}Review}
%Some projects are already mining place (as well as other attributes) from existing data lakes of publication data to provide geospatial and temporal distributions. One such effort is \href{https://www.gdeltproject.org/}{The GDELT Project}, which extracts place as well as actors, sentiment, and event connection (among other elements) from journalistic media across the globe, including publications from as far back as 1979. This and similar projects are powerful and hugely informative, especially as they apply to existing published data. The proposed project should leverage such tools for the inclusion of historic data into the developed database for investigation into the past (already published) incidents. However, the existing automated extraction includes several challenges:
%\begin{enumerate}
%	\item It is not yet perfect, and places may be misattributed (Lisbon, Ohio in the USA may be accidentally attributed to Lisbon, Portugal).
%	\item It does not support the subtlety of incidents occurring in non-conforming places (an incident may not apply to a single administrative boundary but really fall into a subsection of one or several).
%	\item It requires technical prowess and tools to explore the data. A user is unable to define a spatial area of interest (such as their route to work with a half mile buffer or some other irregular shape) and search for all spatially related results, nor it is easy to apply temporal or thematic attributes without prior experience querying results. 
%\end{enumerate}
%Therefore, this project offers a functionality specific to the defined user types of news publication services and provides an appropriate user experience to these.
%
%{5W1H}
%-{\color{orange}“Since events according to our definition occur  at a single point in time, we only retrieve datetimes indicating an extract time, e.g., “yesterday at 6pm”, or a duration, e.g., “yesterday”, which spans the whole day.”\cite{Hamborg2019}}\\
%
%{Rivera2020}\\
%-{\color{orange} Tasks: “1. Get a corpus of news published by the local newspaper through an RSS (Really Simple Syndication) feed, 2. Get a vector characterization based on the well-known Bag-of-Words (Bow) representation, 3. Select features through a mutual information-based method, 4. Train a supervised-learning model, 5. classify online news reports in real-time; the interest is in the ‘traffic accident’ class, 6. process the text of the RSS reports to retrieve the location where the accidents happened, and 7. notify users about the events on a map.”\cite{Rivera2020}}\\
%
%{Lee2019}
%-{\color{orange}Automatic POS extraction: Stanford named entity recognizer, apache open NLP algorithm, MIT information extraction (MITIE), OEDA\cite{Lee2019}}\\
%-Leverated location lists from GEonames, Wikipedia, and Google map\\
%-Preprocessing: building gazetteers and import of data\\
%-Follow for verification and comparison (Validation) type steps if there is time after\\
%
%LOCAL
%
%Crosstown\\
%-{\color{orange}“Our MIssion: Crosstown seeks to deliver community-level data and analyses to the people of Los Angeles who want to make their neighborhoods and city safer, healthier and more connected.”\cite{crosstown2020}}\\
%-{\color{orange} “We want to turn those numbers into stories that people can use to understand what’s happening around them:” \cite{crosstown2020}}\\
%-{\color{orange} Organized by neighborhood boundaries (LA Times), different from administrative boundaries.\cite{crosstown2020}}\\
%-{\color{orange} Leverages public health, crime, administrative inputs, etc.\cite{crosstown2020}}\\
%-{\color{orange} Not geolocating news, but building stories based on location \cite{crosstown2020}}\\
%-{\color{orange} “Crosstown’s focus is not on increasing revenue but decreasing costs of producing local news. It does this through “mass customization” of public datasets in a three-tier system”. \cite{Granger2020}}\\
%-{\color{orange} Geocoded and aggregated by neighborhood\cite{Granger2020}}\\
%-{\color{orange} “We have created hyperlocal news at scale, and also allowed people in the city to see how their neighbourhood fits in with others and understand their experience in context” (- Gabriel Kahn)\cite{Granger2020}}\\
%-{\color{orange} “Tier one: data collection informing news stories”. Collecting and geo-locating “quality of life” data. “Journalists can then access the data and spot newsworthy trends at a glance.'' ``Tier two: data visualisation and dashboards” “There are a lot of different opportunities to take that data and not simply turn it into a classic news story, but turn it into other opportunities for the audience to engage with it.” “Tier three: neighbourhood newsletters”. Avoid the need to have a dedicated reporter in each area.\cite{Granger2020}}\\
%
%- NewsStand
%\begin{itemize}
%	\item {\color{orange} “NewsStand extracts the ‘interesting’ phrases that are most likely to be references to geographic locations and other entities by using NER methods. LOCATION phrases are stored as geographic features of the entity feature vector. Then, it uses a Gazetteer to find those geographic features in the entity feature vector that are names of actual locations. It also employs  Gazetteer to identify the hierarchical information for each location (i.e. country and administrative subdivisions).. After that, it extracts geographic focus (or fous location) based on the frequency of the locations in the news.”  Description of NewsStand used specifically for identifying primary locations of news articles?\cite{Imani2019}}
%	\item {\color{orange}“NewsStand monitors RSS feeds form thousands of online news sources and retrieves articles within minutes of publication. It then extracts geographic content from articles using a custom-built geotagger, and groups articles into story clusters using a fast online clustering algorithm. By panning and zooming in NewStand’s map interface, users can retrieve stories based on both topical significance and geographic region, and see substantially different stories depending on position and zoom level.”\cite{Teitler2008}}
%	\item {\color{orange}NewsStand: Spatio-Textual Aggregation of News and Display \cite{Teitler2008}}
%	\item {\color{orange}Based on ``transactional database technology'' \cite{Teitler2008}}
%	\item {\color{orange} “It places markers representing story clusters on an interactive map interface, thereby allowing meaningful, visual exploration of the news.”\cite{Teitler2008}}
%	\item {\color{orange} “The interplay between significance and zoom level is an important feature of NewsStand, and differentiates is greatly from existing spatially-referenced news reading systems (e.g. the Reuters News Map, the maps locations found in stories using MetaCarta).”\cite{Teitler2008}}
%	\item {\color{orange} Global scalability: obscures articles not of interest to a global audience.\cite{Teitler2008}}
%	\item {\color{orange} Geographic information extraction: “1.Provider scope, the publisher’s geographic location; 2. Content scope, the story content’s geography; and 3. Serving scope, based on the readers’ location.”\cite{Teitler2008}}
%	\item {A global solution \cite{Teitler2008}}
%	\item {\color{orange}“In NewsStand we are also interested in the geographic focus of a collection of news articles about the same subject/topic, rather than just one article, and this is done with the aid of a document clustering algorithm.” \cite{Teitler2008}}
%	\item {\color{orange}“After a new article has been introduced to the system, NewsStand must locate and extract the geographic content from the article. This process, described earlier as geotagging, unifies the explicit textual article content with the implicit geography, and enables spatial exploration of the news.”\cite{Teitler2008}}
%	\item {\color{orange} “Our main goal in designing NewsStand’s user interface was to convey as much geographic and non-geographic information abou current news as possible. The interface consists of a large map on which stories are placed, and the viewing window serves as a spatial region query on the geotagged news stories. Users interact with NewsStand using pand and zoom capabilities to retrieve additional news stories. As users pan and zoom on the map, the map is constantly updated to retrieve news stories for the viewing window, thus keeping the window filled with stories, regardless of position or zoom level.”\cite{Teitler2008}}
%	\item {\color{orange} “NewsStand also features a smaller map that shows the geographic span of the selected story. This minimap allows users to easily see the selected story’s geographic focus, without having to leave the area of interest on the main map.” \cite{Teitler2008}}
%	\item {\color{orange} To avoid geographically clustered visualization of uneven news coverage “Marker selection is therefore a tradeoff between story significance and spread. “To achieve a balance in marker mode, NewsStand divides the viewing window into a regular grid, and requires that each grid square contains no more than a maximum number of markers. The markers to display are selected in decreasing order of story significance and story age. This approach ensures a good spread of top stories across the entire map.\cite{Teitler2008}}\\
%\end{itemize}
%
%Suntimes Crime\\
%-{\color{orange} mapped instances of crime within a city \cite{chicagosun2020}}\\
%-{\color{orange}  “Find out about crime in your neighborhood and your city. The best and most timely analysis of crime trends in the city of Chicago.” \cite{chicagosun2020}}\\
%-{\color{purple}No map or spatial exploration of the news. No textual/tag organization of place.}\\
%
%In Your Area \\
%-{\color{orange} InYourArea lets you follow the latest local news, information, events and more in your area. InYourArea covers local news and information for all Towns, Cities, Villages and Hamlets in the United Kingdom. As well as the latest news and information for where you live, you can also connect with other members of your community, submit events, promote a local business and more.”\cite{InYourArea}}\\
%-{\color{orange} “InYourArea includes news from all the top local and national news publishers and blogs. It includes updates from local councils, police and public services.”\cite{InYourArea}}\\
%-{\color{orange} “InYourArea covers news for all UK Cities, Towns, Villages and Hamlets including London, Birmingham, Manchester, Leeds, Liverpool, Sheffield, Cambridge, Bristol, Norwich, Reading, Cardiff, Edinburgh and more. So download it today and see what's happening in your area. \cite{InYourArea}}\\
%-{\color{orange}Post code based (not searchable by specific locations)\cite{InYourArea}}\\
%-{\color{orange}Method for sharing relevant information by postcode\cite{InYourArea}}\\
%-{\color{orange}Signup: input post code, select three areas of interest\cite{InYourArea}}\\
%-{\color{orange}Includes: state of transit (roads, rail, bus, TFL); weather; notices; recent updates (comments and news stories); all news vs. my news; ability to interact (like, comment, share, follow); no map searching options; whats happening (discussions, about my area, planning applications, funeral notices, council notices) and find (homes near you, items for sale, deals and offers, things to do, local services)\cite{InYourArea}}\\
%
\subsubsection{Manual location extraction from published corpora}
%%-{\color{orange} Document metadata, location based services information extraction, geographical information retrieval\cite{Cai2016}}\\
%%-{\color{orange}“What are the common strategies human use to resolve ambiguity in local spatial references? ANswers to this question will reveal proper decision rules in disambiguation algorithms.”\cite{Cai2016}}\\
%-{\color{orange}Finding 1“There are significant portion of place names in local news that are vernacular, vague places, or finer granularity places that were not found in gazetteers.”\cite{Cai2016}}\\
%-{\color{orange}FInding 2: “THe majority of those place names found matches from gazetteers are found to be ambiguous.”\cite{Cai2016}}11
%-{\color{orange} Finding 3 “For those place names that were not able to find proper footprints in gazetteers, humans can create footprints with ease when supported well.” “Because GeoAnnotator helped human in locating the area and providing good spatial context, drawing the footprint in the map view is relatively straightforward.”\cite{Cai2016}}\\
%-{\color{orange} Finding 4: “There are high degree of verbatim repetition of place references across local news articles over time.” potential “to end up with a local gazetteer that is enriched over time”.\cite{Cai2016}}\\
%-{\color{orange}“heuristic rules that human coders use in disambiguation of place references in local newspaper are very different from those used in resolving ambiguities in global news.”\cite{Cai2016}}\\
%-{\color{orange}“generic gazetteers are not adequate for geoparsing and geomatching for local news.”\cite{Cai2016}}\\
%-{\color{orange}Suggestive geocoding: “Geocoding… requires deep, human-level knowledge to evaluate a potentially large list of candidate matches and make a choice. THese tasks re best done by bringing human and computer to work collaboratively. A simple way to do this is to have a computer doing step 2 (generating all the candidate matches) and uman take care of step 3 (disambiguation). Alternatively, computer can further reduce human efforts in step 3 by ranking the candidates such that those with higher degree of likelihood are placed on the top of the list, and those candidates that are deemed impossible after considering the context are eliminated from the list.” \cite{Cai2016}}\\
%-{\color{orange}“While a system that follows the progressive geospatial referencing framework is expected to enrich its gazetteer and improve its performance over time, ti does require human effort to deal with uncertainties, and complex geographical descriptions that have no gazetteer match (e.g. a long prepositional phrase representing a place). THe key challenge here is to reduce the human effort to the minimal and accelerate the process of incremental improvement.”\cite{Cai2016}}\\
%-{\color{orange}“Geo-annotation requires either local knowledge of the place or general geographic knowledge and the initiative and ability to further research the existence and location of places.”\cite{Karimzadeh2019}}\\
%-{\color{orange}“Presenting place names to users with pre-annotation hastens the geo-annotation process but comes with a risk. It is likely, especially for larger pieces of text, that users might miss place references that are not already highlighted in text by the pre-annotation process, and therefore, these entities are not added to the map view and might be omitted due to visual absence.”\cite{Karimzadeh2019}}\\
%-{\color{orange}“the quality of a geoparser’s output annotation is domain-dependent. Depending on the kind of text (e.g., news story, social media posts or academic articles), the geographic coverage area of documents (e.g., North America, Middle East, Eastern Europe), and the geographic level of toponyms (e.g., country level, state level, physical features, landmarks or building names or addresses), the geoparser may underperform and therefore create additional work for users performing annotation.”\cite{Karimzadeh2019}}\\
%
%
%EXAMPLES:\\
%GeoAnnotator Workbench\\
%-{\color{orange}Protocol: “1. Open an article and read through the whole article to understand the story; 2. Geoparsing: identify all place names (or prepositional phrases) and highlight them in the article; 3. Geomatching and disambiguation: for each identified location reference, search the best gazetteers to find all matches, compare candidates, and finally make a choice from the candidates, and finally make a choice from the candidates. In case no candidate match is found or all candidates are rejected by the human analyst, he or she can create a new footprint and add it to the local gazetteer. 4. Coding: depending on the gazetteer matching outcomes and the disambiguation strategies used by the analyst, each location reference is coded by a case number.” GeoAnnotator workbench\cite{Cai2016}}\\
%-{\color{orange}Uses a local gazetteer (subsect of nominatim), nominatim global, and google maps\cite{Cai2016}}\\
%-{\color{orange}“1. Start with a VGI-based gazetteer that is best for the target locality. The quality of this gazetteer does not matter so much, as it only provides a starting point for us to bootstrap the system. In the current implementation of GeoANnotator+, twe bootstrap the system by taking Nominatim as the initial source of gazetteer. Nominatim provides a search engine API for using OpenStreetMap data as a proxy gazetteer to match place names. The quality of the footprints in Nominatim is quite good, as it normally returns matches of local features by good approximation of their real shape (points, lines, or polygons). We choose Nomination as the initial gazetteer because it is considered the best VGI-based geographical data sources with good coverage in most localities worldwide. 2. Progressive geocoding to enrich gazetteer incrementally. Provide a semi-automated workbench for human analysts to evaluate all matches and make a choice. If none of the matches are appropriate, analysts will create the footprint in the workbench. The system not only remembers the result of this human-generated geocode, but also adds a new entry to its gazetteer as enrichment. In this way the system is able to leverage real local and community conversations and outsource this geocoding task to local community members. Hence, new gazetteer entries learned from human reflect local spatial language and local knowledge. 3. Smart footprint recommender rankst he matched candidates by their likelihood of being correct. Based on our understanding of how humans disambiguate multiple interpretations of place names, a computer can play the role of a smart recommender by automating a set of heuristic rules. By presenting the candidate list in the order of likelihood, human annotators are more likely to find the answer by exploring only the top few candidates. This creates savings to human cognitive efforts.” 1 and 2 are critical (but with two- how should they be saved/named when hyper specific?), and 3 is gravy.\cite{Cai2016}}\\
%-{\color{red}Smart recommendation: “prioritize previous annotations”, “exploit local gazetteer”, “rank search results with regard to a moving focus”. See p7 fr more.\cite{Cai2016}}\\
%
%\subsubsection{Location assignment}
%During publication\\
%-Geo My WP: {\color{orange} Integrates GoogleMaps, Leaflet, and OpenStreetMaps, includes proximity search, point location definition \cite{Fitoussi}}\\
%- CG Geo Plugin: {\color{orange}  Focused on customer geolocation integration: geotargeting, legal requirements, etc. \cite{CFGeoPlugin}}\\
%
