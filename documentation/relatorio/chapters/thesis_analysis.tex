\subsection{Future opportunities}
\subsubsection{Validation}
Though testing of these hypotheses through rigerous comparison to the status quo (traditional online news sources without a spatial element) and emerging product performing automatic extraction of place (such those of the GDELT project) are not included in this endeavor, the resulting tools should provide a basis from which future projects may develop and evaluate.

\subsubsection{Application to GeoIntelligence}
-{\color{purple} Potential future use of this toolset: GeoINT source.\cite{Datta2018}}\\
-{\color{purple}Potential future focus/further development for dashboard monitoring OR the system could feed into such systems, providing high confidence georeferencing as it has been manually defined by the author.\cite{Teitler2008}}\\

\subsubsection{Sub-article definitions}
-{\color{orange}Future research directions: “spatial role labelling… ‘the task of identifying and classifying the spatial arguments of spatial expression mentioned in a sentence’... spatial role labelling is key not only in geographic information retrieval but also in domains such as text-to-scene conversion, robot navigation or traffic management systems.”\cite{Karimzadeh2019}}\\
-{\color{orange}“We should experiment with how to visualize uncertainty, possible errors and imperfections in our data. And most importantly, we should keep in mind how data can be a powerful tool for all designers, bringing stories to life in a visual way and adding structural meaning to our projects.”\cite{Lupi2017}}\\

\subsubsection{Recommended extraction of place}
-{\color{purple}Avoid the issue of high volume evaluation all together. Those who write the story can apply the location which is the most accurate option. In the absence of the author assigning place, automatic extraction (perhaps such an LSTM model) is an appropriate tool for historical articles.\cite{Halterman2019}}\\

\subsubsection{Incorporation of historical stories}
GDELT

\subsubsection{Statistical analysis}
-{\color{orange} Online clustering: “a clustering algorithm for the news domain should group together all news articles that describe the same news event into groups of articles termed story clusters. Broadly, a news event is defined in terms of both story content and story lifetime - articles in the same cluster should share much of the same important keywords, and should have temporally proximate dates of publication.”\cite{Teitler2008}}\\
-{\color{orange}Future development: “We will consider ways to use clustering to determine the news provider’s geographic scope (i.e. the geographic location of the newspaper), and use it to improve both geotagging and local news coverage.” }{\color{purple}leveraging local knowledge/gazetteers}\cite{Teitler2008}\\

