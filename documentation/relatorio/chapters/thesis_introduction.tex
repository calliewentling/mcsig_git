The ongoing COVID pandemic has highlighted the value of the visualization of information on a map, not only for specialists to monitor and predict viral outbreaks, but to arm the public with empowering information as well. Of course, the value of geographic information systems (GIS) goes beyond public health services and is already nestled into our everyday activities in the form of daily tasks such as navigation and service selection. Applications like Google Maps, AirBnB, and UberEats allow non-technical users to visualize and filter the distribution of various services through spatial (SA), temporal (TA), and thematic attributes (ThA). For example, a user on AirBnB may filter all apartments with high
speed WiFi (ThAs) available in the Estrela neighborhood and within walking distance to a market (SAs) from Aug 1 to Aug 7, 2020 (TA).

Yet, though this type of manipulation is commonplace in the products of many industries, it is glaringly absent from that of news media. When reading about an incident occurring in an unfamiliar place, readers will often need to look up the location. They may have trouble relating the spatial significance of an incident to neighboring occurrences or historical events in the same spot. Many articles define place via textual descriptions, but these can be easily overlooked if searched by keyword, especially if different names or alternate designations are employed by the searcher. This is a problem for researchers who may want to define a study area that does not conform to traditional administrative boundaries or existing points of interest, but also for the casual user or city official. The former might, while perusing headlines, miss an article of interest relating to a place along their commute home from work. The latter could be an elected official who seeks to monitor an issue (such as gentrification or homelessness) but is unable to visualize the subtle distribution of such events throughout his or her district. In these cases, as well as a host of others, there is obvious disconnect between the existence of data and its usability. Though many search engine queries contain geographic keywords \cite{Silva2006}, news media enterprises have not yet accommodated such spatial associations to their articles that would provide an expected improved user experience and therefore competative edge in their industry. As such, there is commercial as well as well as operational and academic value in better understanding the spatial distribution of events within a community, such that additional informative insights can be drawn.

This project seeks to develop a set of functional tools that supports the creation and management of a spatial database of local news stories, a publishing interface (associating place and adding records to the database), a user interface (list and map format search, filter and visualization of results from the database), as well as a story visualization plugin (a map displaying the distribution of a story in a contextual map per story page), see Appendix \ref{appendix:organization}. This proof of concept (POC) functionality should demonstrate the value of new spatial products in news media, and provide a basis from which meaningful projects may be developed for mass media applications in the future.

%-{\color{orange} “Geographic entities are frequent in user queries. Approximately 4\% of the queries logged by tumba! contain the names of the same 308 municipalities. If we considered names of localities, landmarks or streets, this percentage would increase. It would certainly also increase if our engine considered geographic semantics and proximity when giving results. Other studies have also shown that geographically related queries represent a significant sub-set of the queries submitted to a global search engine.”\cite{Silva2006}}\\