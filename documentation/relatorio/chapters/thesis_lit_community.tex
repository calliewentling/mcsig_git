\subsection{Citizen empowerment}

\subsubsection{Smart communities}
Though most often referred to on the city level, smart communities leverage information and communication technologies (ICT) (networks) and various sources of data (sensors) to address and improve the functional needs of its population (actuators), engaging its ``users'' to develop citizen-centered interventions and responding to their changing needs\cite{Williams2016,Roche2012, Afzalan2017}. %{\color{orange}“A smart community may be one of any size or significance, geographically separate or part of some larger urban unit, that employs the IOT to: improve aspects of its operations or other factors within or outside its boundaries that are important to its economic vitality, safety, environmental footprint, quality fo life or other factors deemed significant; respond to the community’s changing needs rapidly and efficiently; engage the community and enable informed understanding of, and where applicable consent to, what it is doing; collaborate with other communities as needed or desired.”\cite{Williams2016}};{\color{orange}“A smart city is roughly described as a platform or a system of systems, essentially based on three components: Sensors, Networks and Engagement (actuators).” \cite{Roche2012}.}; {\color{orange}“Information and Communication Technologies (ICTs) have given rise to the ideal that cities will become increasingly smart, connected, responsive, and citizen-centric”.}\cite{Afzalan2017}\\
In fact, {“[a]n active and engaged citizen is indeed the main driving force of a `smart city'''\cite{Oliveira2021}}. Though smart cities also address economic vitality and environmental impact in addition to social well-being, empowered communities increasingly expect the ability to influence their environments, such by affecting goverments planning procedures and services\cite{Williams2016}. %-{\color{orange}Smart community: “Communities can be large or small, and they may or may not be ac enter of some noteworthy aspect of human activity; they may be separated from other communities or they may be aggregated into a conurbation of some kind. But any community of any size or significance can be ‘smart’. Communities should also define the services of interest to them, within their territorial boundary or outside it, as required.”\cite{Williams2016}}; -{\color{orange}“there are dimensions of smart which the community pursues.”\cite{Williams2016}}\\
Beyond efficiency, citizens require safer, more enjoyable living experiences in all aspects of their lives. Governments may accommodate the public interest\cite{Afzalan2017}  %-{\color{orange}“Planning and decision making is not all about efficiency, but also about responding to the public interest and community needs.”}\cite{Afzalan2017}\\
by incorporating four dimensions (intelligence, digital, open, and live, referring to its social and informational infrastrucutres, open governance, and continuity of adaptation, respectively) of  smart communities\cite{Oliveira2021}. %- {\color{orange}“There are four dimensions in which a smart city primarily operates, names: intelligent city (its social infrastructure), the digital city (informational infrastructure), the open city (open governance) and the live city (a continuous adaptive urban living fabric) \cite{Oliveira2021}}.\\
The identification and monitoring of community dynamics requires ``sensing life'' though open dialogues with constituents as well as internet of things (IOT: the integration of networked hardware sensors to monitor and/or interact with their surroundings) technology\cite{Roche2012}. %{\color{orange}“Making a city smarter is neither only a technological infrastructure issue, nor a managerial one. It is also and essentially providing citizens a better and safer way of living in urban areas, in teh places where they live, work, have fun, consume… Therefore, sensing life in those places is a major stake to understand new city dynamics and then to design better living urban environments.”} \cite{Roche2012}\\
This sensing infrastructure leverages various sources of data to determine the state of various subsystems and support interventions for improvement. %- {\color{purple}Leverages various sources of data to determine the state of various subsystems and support interventions for improvement.}  \\
Ideally, it may identify potential opportunities for improvement but is more commonly leveraged in application focused scenarios, in which a ``search, evaluate, and process'' method is employed in response to a particular challenge\cite{Jiang2020}. %{\color{orange}“In the big data era, the typical scenario does not start from data but from application. THe research procedure is shifting to a new scenario - ‘search, evaluate, and process’ -, that is, identifying the applications, discovering all data related, adn running a processing model for knowledge generating.”\cite{Jiang2020}}\\

The information age itself is a source of both challenges and potential solutions. 
Since the turn of the century, all facets of urban life and the structures that support them have transitioned towards the digital and informational, .% -{\color{orange}“DUring the last two ecads, urban structures have become more digital and information-based, and there has been a decisive change in the living environment of citizens.”\cite{Rivera2020}}\\
A community as "a system of systems"\cite{Roche2012} %{\color{orange}“A smart city is roughly described as a platform or a system of systems, essentially based on three components: Sensors, Networks and Engagement (actuators).” \cite{Roche2012}.}
has an internal structure\cite{MasseyD1991}, %{\color{orange}“‘Communities’ too have internal structures.”}\cite{MasseyD1991}\\
with corresponding spheres of influence of its nodes within and outside of these.  
As quickly as informational technology (IT) tools provide new means of characterising the immediate, physical geographic area of a community node, it also supports the digital transmission of ideas and participation to remote parties via direct communication platforms as well as the more public arenas of social media. In short: "the geography of social relations is changing"\cite{MasseyD1991}, %-{\color{orange}“the geography of social relations is changing. In many cases such relations are increasingly stretched out over space. Economic, political and cultural social relations, each full of power and with internal structures of domination and subordination, stretched out over the platen at every different level, from the household to the local area to the international.”}\cite{MasseyD1991}\\
with digital connections offering  "unique opportunities to identify and understand information dissemination mechanisms and patterns of activity in both the geographical and social dimensions, allowing us to optimize reponses to specific events"\cite{Oliveira2021}.%- {\color{orange}“As the popularity of social media is growing exponentially we are presented with unique opportunities to identify and understand information dissemination mechanisms and patterns of activity in both the geographical and social dimensions, allowing us to optimize responses to specific events," \cite{Oliveira2021} }
Data in general is already highly regarded as a key comodity for developing an economy\cite{Lupi2017}, % -{\color{orange}“Data is now recognized as one of the founding pillars of our economy, and the notion that the world grows exponentially richer in data every day is already yesterday’s news.”\cite{Lupi2017}}\\
To harness the value of this ever-expanding resource, community operations should accommodate methods for capture, exploring, and sharing this data, spatial or otherwise, and its processed results\cite{Roche2012}.% - {\color{orange}``the need of an enabling geospatial information platform to facilitate data discovery and access in order to support smart cities’ operations.” \cite{Roche2012}.}\\
Beyond operational efficiency, the information products and services have the potential to stimulate new creative uses that facilitate the economic, social, and environmental well-being of the participants of the community\cite{Rajabifard2009}. %-{\color{orange}“The creation of economic wealth, social stability and environmental protection in line with MDGs can be achieved through the development of products and services based on spatial information collected by all levels of government. These goals and objectives can be facilitated through the development of a spatially enabled government and society, where location and spatial information are regarded as common goods made available to citizens and businesses to encourage creativity and product development.” MDG: Millennium Development Goals}\cite{Rajabifard2009}\\
Just as the context of a community -- its culture, history, environment, access to technology, demographics, etc. -- can vary tremendously across time and space, so too should its interventions\cite{Afzalan2017}. %-{\color{orange} “The issue of context-sensitivity may require more attention from planning organizations that try to choose new technologies. Cities are overwhelmed with the availability of new communication technologies and the opportunities that the technologies offer. This issue also exacerbates with the social pressure they receive from citizens who expect the implementation of smarter governance systems.”}\cite{Afzalan2017}\\
Members of such knoweldge societies\cite{Rivera2020} %-{\color{orange} Smart city and knowledge society.\cite{Rivera2020}}\\
, investigators and entrepreneurs or anyone with with access to technology, are better equiped to address local psychographics ("the prevailing interests of people in an area"\cite{Chiappinelli2020} %-{\color{orange}Psychographics: “the prevailing interests of people in the area.”}\cite{Chiappinelli2020}\\
in nontraditional or niche applications\cite{McQueenBaker2019}.  %-{\color{orange}“Increased accessibility of technology pushes researchers to consider how these tools might be used in critical scholarship in nontraditional ways.”\cite{McQueenBaker2019}}\\
Such opportunities can even unburden institutions with the responsibility of managing, processing, and transforming data into relevant services, and instead allow the community itself to develop novel applications for public resource that can be adapted into operations when mature.

%%%%%%%%%%%%%%%%%%%
%- Unlike traditional polling, which attempts to gauge the public pulse on a variety of political and social issues, social media platforms and forums allow the unsoliscited, unformatted presentation of personal perspectives and permit the public evolution of dialogue around these ideas. {\color{purple}News reporting as a more official or pre-aggregated public pulse can be layered in as well}. \cite{Oliveira2021}.\\
%The nature of this digital discourse allows the collection not only of publically shared oppinions, but often the
%- {\color{orange}while the identification of hotspot emergence helps us allocate resources to meet forthcoming needs”.} \cite{Oliveira2021} 
%%%%%%%%%%%%%%%%%%%



\subsubsection{Communication}
%COMMUNICATION
In the course of its operations, a smart community should facilitate a "shared understanding of what is happening" within it\cite{Rivera2020},%-{\color{orange} A smart community should (among other things) “Enable a shared understanding of what is happening in the ‘city’.” via “the need to notify citizens about traffic accidents”\cite{Rivera2020}}\\
from planned works to unforeseen incidents. Just as big ideas are evolving through digital channels, so too has the sharing of neighborhood news gone online. Phsyical proximity is no longer the primary means of passing the latest hearsay. Words are leapfrogging the traditional stoop-to-stoop transmission and sharing information via networkng platforms\cite{Evans-Cowley2010}. %-{\color{orange} “In today’s world, words are moving rapidly, allowing neighbors to share news that was once passed along porches or stoops. Citizens are increasingly sharing information via social networking and virtual reality tools, rather than from the front porch.”} \cite{Evans-Cowley2010}\\ 
Following suit, many news channels and government communication departments have incorporated digital distribution strategies, often that leverage social media to engage readers and direct traffic to their channel paltforms. This allows not just local eyes on local announcements, but also invites remote viewers to participate\cite{Evans-Cowley2010}. %-{\color{orange} dIgital tools support engagement beyond physical borders} \cite{Evans-Cowley2010}\\

% DATA VISUALIZATION
Stemming from the assumption that "storytelling is the most effective way to merge meaning and emotions"\cite{WEF2021}, %-{\color{orange}“Storytelling is the most effective way to merge meaning and emotions.”\cite{WEF2021}}\\
a tremendous and increasingly more ubiquitous tool for effective and relatable communication is data visualization\cite{Lupi2017,storiesGL}. %-{\color{orange} “Data, if properly contextualized, can be an incredibly powerful tool to write more meaningful and intimate narratives.”\cite{Lupi2017}}\\ -{\color{orange} “Collecting data is a way for us to build records and preserve memories. Reclaiming the human approach to data allows for narrative components that can act as very powerful tools and design materials to create stories we can all relate to.” Giorgia Lupi' \cite{storiesGL}}\\
Data visualization products and inclusions have migrated beyond the niche tech or empresarial applications to "a part of the fabric that is modern culture", threading their way into newspapers, fasion lines and books\cite{Meeks2019}. %-{\color{orange}“Data visualization is becoming less of a tech company rarity and more a part of everyone’s everyday life… it will only continue to grow more common in the coming years.” \cite{Meeks2019}}\\ %-{\color{orange}“Modern data visualization is optimized for producing charts for busy executives. But that’s changing. Now, data visualization is personal stories, small businesses, data science, political campaigns, human resources, community building -- in short, data visualization is becoming a part of the fabric that is modern culture.”\cite{Meeks2019}}\\ % -{\color{orange}“2019 saw the United States President amend a data visualization product with a sharpie. That should have been enough to make 2019 special, but the year also saw the introduction of a data visualization-focused fashion line, a touching book that uses data visualization to express some of the anxieties and feelings we all struggle with, as well as the creation of the first holistic professional society focused on data visualization.”\cite{Meeks2019}}\\ %-{\color{orange}“public debates about the presentation of data increase the prominence of data visualization as a meaningful act.” \cite{Meeks2019}}\\
Studies indicate that readers prefer pictoral and summarial forms of information (as opposed to purely textual)\cite{Evans-Cowley2010}. %-{\color{orange}“visually prominent and summary-based information, such as maps and pictures, are participants’ preferred form sof information.” }\cite{Evans-Cowley2010}\\ 
Visuals can provide additional context, identify changes, reveal patterns, and showing and distinguishing between relationship types\cite{Shneiderman1996},  %-{\color{orange}“Visual displays become even more attractive to provide orientation or context, to enable selection of regions, and to provide dynamic feedback for identifying changes”.\cite{Shneiderman1996}}\\ %-{\color{orange}“Abstract information visualization has the power to reveal patterns, clusters, gaps, or outliers in statistical data, stock-market trades, computer directories, or document collections.”\cite{Shneiderman1996}}\\ %-{\color{orange}“The attraction of visual displays, when compared to textual displays, is that they make use of the remarkable human perceptual ability for visual information. WIthin visual displays, there are opportunities for showing relationships by proximity, by containment, by connected lines, or by color coding.”\cite{Shneiderman1996}}\\
ultimately "connect[ing] numbers to what they really stand for: knowledge, behaviors, people"\cite{Lupi2017}. % -{\color{orange}“We are ready to question the impersonality of a merely technical approach to data and to begin designing ways to connect numbers to what they really stand for: knowledge, behaviors, people.”\cite{Lupi2017}}\\
Users,  whether they be the general public or decision makers, are expected to have some data visualization literacy (DVL)\cite{Borner2019}. %-{\color{orange} “In the information age, the ability to read and construct data visualizations becomes as important as the ability to read and write text.”\cite{Borner2019}}\\ 
This mutual expectation of information producers, consumers, and actors to present and ingest efective representationsf is reenforcing its importance and creating new standards of competencies.  %-{\color{orange}“The invention of the printing press created a mandate for universal textual literacy; the need to manipulate many large numbers create the need for mathematical literacy; and the ubiquity and importance of photography, film, and digital drawing tools posed a need for visual literacy. Analogously, the increasing availability of large datasets, the importance of understanding them, and the utility of data visualizations to inform data-driven decision making pose a need for universal data visualisation literacy (DVL).” \cite{Borner2019}}\\
"Every publisher and journalists knows the value of charts and wants more of them"\cite{DNIFund2018}, not only for aesthetic breaks in text blocks and their power to convey complex information memorably, but also the jumps in page views that they generate\cite{Meeks2019,storiesGL}. %-{\color{orange}“Every publisher and journalist knows the value of charts and wants more of them. Done well, charts throw light on difficult topics - they make stories easy to read by breaking up oceans of text. The reality is different. Busy newsrooms don’t always have the tools or skills to create great charts. And in a world of daily, if not hourly, deadlines, they often don’t have the time.”\cite{DNIFund2018}}\\ %-{\color{orange}“As well as improving reader understanding of complex issues, the charts are generating 5.6\% more page views.”\cite{DNIFund2018}}\\ %-{\color{orange}In 2019, “data visualization featured prominently in major new stories and key players in the field created work that didn’t just do well on Dataviz Twitter but all over.”\cite{Meeks2019}}\\ %-{\color{orange}“Data is increasingly important to all businesses, not just tech, and so much a part of our everyday lives that it makes sense that companies with strong data analysis, data science, and data engineering talent would feel the need to improve their data visualization capabilities.” \cite{Meeks2019}}\\ %-{\color{orange} “Data projects can be visually beautiful, but their true power lies in their ability to convey meaningful and relatable narratives about parts of the world and people that we can’t necessarily observe.” Giorgia Lupi\cite{storiesGL}} %-{\color{orange}“Both SalesForce and Google have already invested significantly in their own in-house data visualization tools but both realized that to compete they needed to rapidly expand their data visualization capacities and were willing to pay top dollar to do so.”\cite{Meeks2019}}\\
Simultaneously, academia has established a variety of digital visual literacy frameworks (DVL-FW) that are being adopted and taught from primary school through higher and continuing education programs. %Data visualization literacy framework (DVL-FW)

However, beyond the ability to create compelling visuals to contextualize or communicate important information is the discernment to understand the different insights needed by each stakeholder\cite{Borner2019}. %-{\color{orange}Insight needs: “DIfferent stakeholders have different insight needs (also called “basic task types”) that must be understood in detail to design effective visualizations for communication and/or exploration.”\cite{Borner2019}}\\
Data visualization can represent answers to the questions like \textit{who}, \textit{what}, \textit{when}, and \textit{where} by incorporating different kinds of representations (network, topical, temporal, and geospatial analysis, respectively). % -{\color{orange}Visualization types: “statistical analysis (e.g., to order, rank, or sort); temporal analysis answering “when” questions (e.g., to discover trends); geospatial analysis answering “where” questions (e.g., to identify distributions over space); topical analysis answering “what” questions (e.g., to examine the composition of text); and relational analysis answering “with whom” questions (e.g., to examine relationships; also called network analysis).”\cite{Borner2019}}\\
Maps increasingly being employed to answer \textit{where} and related questions as they are more easily interpreted and remembered\cite{Borner2019}. %-{\color{orange}“Controlled laboratory studies examining the recall accuracy of relational data using map and network visualizations have found that map visualization are easier to read and increase memorability”.\cite{Borner2019}}\\
They illucidate spatial relationships using layers of data contextualized by basemaps (such as raster images or vector representations)\cite{Jiang2020,McQueenBaker2019}. %-{\color{orange}“Data visualization is another important functionality for communicating geospatial information to users. A popular method for visualizing data is to use an online map for allowing users to visually evaluate a dataset. Online maps can be interactive allowing panning and zooming and possibly changing the visual appearance of the base map (e.g. satellite image, vector map). Geoportals can also provide data download functionality providing either the data directly or through sharing of dataset links.”\cite{Jiang2020}}\\ %-{\color{orange}“Maps then illustrate layers of data, visually displaying spatial relationships.”\cite{McQueenBaker2019}}\\
Especially when presented digitally, online maps (much like other charts) provide the opportunity for dynamic exploration and additional insight by visual inspection of elements and their spatial or thematic relations to each other. Especially when evaluating foreign areas, maps can provide especially valuable context by concisely representing proximities and directional situations, versus relying on verbal descriptions that may perhaps be more easily misconstrued\cite{Shneiderman1996}. %-{\color{orange}“Exploring information collections becomes increasingly difficult as the volume grows. A page of information is easy to explore, but when the information becomes the size of a book, or library, or even larger, it may be difficult to locate known items or browse to gain an overview.”\cite{Shneiderman1996}}\\ %-{\color{orange}“A picture is often cited to be worth a thousand words and, for some (but not all) tasks, it is clear that a visual presentation - such as a map or photograph - is drastically easier to use than is a textual description or a spoken report.”\cite{Shneiderman1996}}\\

In any case, data should be used and interpreted cautiously. Data records are an abstraction of the real world\cite{Lupi2017}. %-{\color{orange} “Let’s just stop thinking data is perfect. It’s not. Data is primarily human-made. ‘Data-driven’ doesn’t mean ‘unmistakably true,’ and it never did.”\cite{Lupi2017}}\\
Often, visualization of data disclude elements of uncertainty\cite{Meeks2019} %-{\color{orange}“The naive perspective that data visualization is just a final step to help people see the data ignores the importance of subtle steps like showing uncertainty as well as the necessity to design a product that engages the audience”\cite{Meeks2019}}\\ 
or are developed prematurely (without proper analysis) or improperly (misleadingly)\cite{Borner2019,Monmonier2018}.  %-{\color{orange}“Most datasets need to be analyzed before they can be visualized”\cite{Borner2019}}\\ %None for Monmonier
Further, visualization designer may overestimate the ability of the consumer to interpret them quickly and accurately -- one must be careful to display images that can be ingested as intended, without sacrificing nuance of complex issues when it is critical for decision makers\cite{Borner2019,Lupi2017,Zhang2019}. %-{\color{orange}“the time required to read a visual image increases systematically with the distance between initial focus point and the target - independent of the ‘amount of material’ between both points.”\cite{Borner2019}}\\ %-{\color{orange}“However, when subjects had to perform computation while reading a visualization, comprehension became more difficult, showing that the interpretation of graphs is ‘serial and incremental, rather than automatic and holistic’.” \cite{Borner2019}}\\ %-{\color{orange}Because of the “iterative nature of graph comprehension”, “the importance of spatial processes (e.g., the temporal storage and retrieval of an object’s location in memory, allowing for mental transformations, such as creating and transforming a mental image) for the graph comprehension.”\cite{Borner2019}}\\ %-{\color{orange}“the more usage and actionable insights gained, the more important it becomes to empower individuals to properly construct and interpret that visualization.”\cite{Borner2019}}\\ %-{\color{orange}“disparate juxtaposed visualizations are insufficient for understanding nuanced temporal interactions between many data sources.”\cite{Zhang2019}}\\ %-{\color{orange}“The phenomena that rule our world are by definition complex, multifaceted and mostly difficult to grasp, so why would anyone want to dumb them down to make crucial decisions or deliver important messages?”\cite{Lupi2017}}\\ % -{\color{orange}“Prior research on DVL shows that people have difficulties reading most visualization types but especially, networks.” \cite{Borner2019}}\\ %-{\color{orange} “Data doesn’t have to be scary or intimidating, because, if you think about it, data isn’t even real! Rather it’s abstract, representing the details of our lives and our ideas.” Giorgia Lupi'\cite{storiesGL}}\\

%%%%%%%%%%%%%%%%%%%%%%%%%%%%%
%-{\color{orange}“We can write rich and dense stories with data. We can educate the reader’s eye to become familiar with visual languages that convey the true depth of complex stories.”\cite{Lupi2017}}\\
%-{\color{orange}“Like other literacies, DVL aims to promote better communication and collaboration, empower users to understand their world, build individual self-efficacy, and improve decision making in business and governments.”\cite{Borner2019}}\\
%-{\color{orange}“Overall, the bandwidth of information presentation is potentially higher in the visual domain than for media reaching any of the other senses.”\cite{Shneiderman1996}}\\
%-{\color{orange}“These interactive data visualizations inform the public and guide decision-makers to save lives”.\cite{Shneiderman2020}}\\
%-{\color{orange}“Data visualization as a way of exploring and expressing one’s feelings and traits has always been present in the margins of the field.”\cite{Meeks2019}}{\color{red}Data-driven badges\cite{Meeks2019}}\\
%-{\color{orange}  “But we’re now seeing an acknowledgement that if you don’t have good data visualization then your insights are less apparent, resonate less with audiences and are harder to communicate among scientists.”\cite{Meeks2019}}\\
%-{\color{orange} bilingual map legend example \cite{Witschas2004}}\\
%-{\color{orange}Type by task taxonomy (TTT)\cite{Shneiderman1996}}\\
%Example:\\
%%%%%%%%%%%%%%%%%%%%%%%%%%%%%%%%%%%%%%%%%






\subsubsection{Public participation}
Public participation is a critical element of citizen empowerment. democratic vibrancy, and innovation\cite{Afzalan2017}.%. “smart-city approaches should contribute to innovation and enhance democratic decision making and transparency through public participation.”}\cite{Afzalan2017}\\ %-{\color{red} “The importance of encouraging people to act as participative citizens in issues of public concern is essential for a functioning democracy, particularly when researchers are observing that civic engagement (CE) is diminishing in developed countries”} \cite{Acedo2019}\\ %-{\color{orange}“Often-cited benefits of participation include increasing the education and awareness levels of the citizenry, civic engagement, government responsiveness, and citizens’ commitment to implementation”.} \cite{Evans-Cowley2010}\\
It provides opportunities for citizenry to provide feedback on services and provide new ideas based on lived realities, but also opportunities for collaboration and motivated co-productions with interested, non-institutional stakeholders within the area\cite{Acedo2019}. %-{\color{orange} CE: Civic engagement; “ ways in which citizens have a common purpose of preserving and promoting public goods, to improve conditions for others, community or collective benefit.” } \cite{Acedo2019}\\ 
Further, participation strengthens a community by building social capital amongst its participants, demonstrating trust between members\cite{Evans-Cowley2010}. %-{\color{orange}“Participation helps to build social capital in a community, which in turn strengthens the community.”} \cite{Evans-Cowley2010}\\
High forms of civic engagement (CE) assume that citizens have the power to influence decisions that will touch their own lives, whether through active dialogues or other means of engagement. %-{\color{orange} “The highest level of participation opportunities hold that all citizens must be equally empowered and fully informed to ensure that they can exert influence in decisions that affect them”“The highest level of participation opportunities hold that all citizens must be equally empowered and fully informed to ensure that they can exert influence in decisions that affect them”}\cite{Evans-Cowley2010}\\ -{\color{orange}Citizens engagement: “invites coproduction of content without necessarily engaging contributors in dialogue”.}
Thoug not a new concept, today's communities are more and more expecting that relevant organizations will provide opportunities for such feedback, which (if implemented appropriately) may harness public knowledge for the better of said organiation and the community as a whole. %-{\color{orange} “The rate of utilization and willingness to accept new methods may be based on teh demographics of a community. In some communities, there is a growing expectation on the part of citizens that there will be online participation opportunities.” }\cite{Evans-Cowley2010}\\ %-{\color{orange}“An active particpatory environment that uses Internet technology has the potential to engage the public, and may therefore facilitate knowledge retention and use by the public.”} \cite{Evans-Cowley2010} \\ 5-{\color{orange} “Public participation has been a hallmark of the planning process for over 30 yeras, with each generation trying to improve access and interactivity to ordinary citizens.”} \cite{Evans-Cowley2010}\\
Updated strategies, especially those that include presentially and digitally hybrid participation options, may engage larger audiences, facilitating greater particpiation while mitigate possible digital divides in participating demographics\cite{Evans-Cowley2010, Afzalan2017}. %-{\color{orange} “Such new and creative participation strategies [hybrid: partially in-person and online experiences] could ameliorate the potential harmful effects of an increasingly digital divide.”}\cite{Evans-Cowley2010}\\ %-{\color{orange}“Using online tools in public meetings can facilitate the participation of a more diverse community.”}\cite{Afzalan2017}\\
It can also prolong interactions, allowing all stakeholders to reevaluate options and motivations throughout the entire process\cite{Afzalan2017}. %-{\color{orange}“THese processes focus on responding to public interest and promoting open-ended interactions to provide opportunities for participants that constantly redefine the “what” and “how” of the issues they address. THese processes can provide opportunities for consensus building or learning among diverse stakeholders, democratic decision making, mobilizing actions, engaging local knowledge, or responding to regulations or community norms.” }\cite{Afzalan2017}\\ %-{\color{orange}“Today, technology allows for an entirely new generation of forms and practices of public participation that promise to elevate the public discourse in an unprecedented manner while providing an interactive, networked environment for decision-making.”}\cite{Evans-Cowley2010}\\
These services may be government or institutional services (ex: Lisboa Participativa), non-institutional platforms (CitySourced), or commercial products leveraged for engagement (ex: NextDoor). %-{\color{orange}“ONline participatory tools (OPTs) refer to two types of technologies: (1) web-based tools that are particularly designed for public engagement (e.g. MySideWalk, PlaceSpeak, CitySourced, Crowdbrite); and (2) social networking sites (e.g. Facebook, NextDoor) that are not designed for public engagement but can be used for participatory planning.” \cite{Afzalan2017}\\

As an "inherently spatial" element, public participation should not be disconnected from this dimension\cite{Acedo2019}. %-{\color{orange}“CE and participation are inherently spatial and, consequently, influence by social relations, time and space. THe spatial dimension of CE (e.g., planning decisions or decision-making processes about communal spaces) has been established in administrative boundaries because of the availability of census and socioeconomic data in those areas. However, teh approach has probably hidden the spatial nature of CE associated with space, place and locality - essential characteristics to determine who is interested in the participatory processes and why.”} \cite{Acedo2019}\\
Spatial information is critical for making educated decisions on key human issues \cite{Rajabifard2009}.%-{\color{orange}“Ready and timely access to spatial information - knowing where people and assets are - is essential and is a critical tool for making any informed decisions on key economic, environmental and social issues.”}\cite{Rajabifard2009}\\. 
Though this clearly applies to decisionmakers in their respective fields, access to location services is must be a given for all modern society\cite{Rajabifard2009}. %-{\color{orange}“Spatial informaiton is an enabling technology /infrastructure for modern society.”}\cite{Rajabifard2009}\\
 "[A] city could not be smart without spatially enabled citizens"\cite{Roche2012}, %- {\color{orange} “a city could not be smart without spatially enabled citizens.”} \cite{Roche2012}\\ 
who are able to contextualize their own experiences and needs in relation to the realities of their peers.
In 1996, this was recognized by the National Center for Geographic Infromaiton and Analysis in the United States of America which establisehd the public particiaption geogrpahic information system (PPGIS) to better accommodate marginalized populations \cite{Brown2012}. %-{\color{orange}“The term ‘public participation geographic information systems’ (PPGIS) was conceived in 1996 at the meeting of the National Center for Geographic INformation and Analysis in the United States to describe how GIS technology could support public participation for a variety of applications with the goal of greater inclusion and empowerment of marginalized populations.”\cite{Brown2012}}\\
It can be especially powerful to visualize the impact of interventions of underrepresented communities at scale\cite{McQueenBaker2015}. %-{\color{orange} “Mapping technology has also been included in community engagement projects and sites of activism, leading to a new form of visual politics. Scaled representation of the community has helped stakeholders understand political implications for their community in a broader context.”\cite{McQueenBaker2019}}\\
At its core, a PPGIS represents an abstract of thematically interesting features, contributing to a communal understanding of place\cite{Brown2012}. %-{\color{orange}“Common to all types of PPGIS data capture is the need to symbolically represent the spatial attribute of interest on a map.”\cite{Brown2012}}\\ %-{\color{orange}“PPGIS combines the practice of GIS and mapping at local levels to produce knowledge of place.”\cite{Brown2012}}\\
"[I]n spite of all the technological developments in recent years, one of the biggest barriers to public participation in urban policies remains unsurpassable: the difficulty that people have to understand how the planning proposals are projected in space, how they redefine it, and how they impact the use of urban space\cite{Painho2013}. %-{\color{orange} “in spite of all the technological developments in recent years, one of the biggest barriers to public participation in urban policies remains unsurpassable: the difficulty that people have to understand how the planning proposals are projected in space, how they redefine it, and how they impact the use of urban space.”} \cite{Painho2013}\\
This kind of technology supplements top down and bottom up activism to provide a common foundation from which to build collaborative understanding and develop effective interventions through "active citizenship" \cite{Kleinhans2015}.%-{\color{orange}“Active citizenship, promoting citizens’ self-organization and engagement in urban development are high on the political agenda. As a result, many cities are experiencing a surge in place-based and technology initiatives, both government-initiated and grassroots activism, and collective networked action to foster civic engagement in urbn and neighbourhood contexts.”}\cite{Kleinhans2015}\\
From this, such online tools should include elements of understanding the decision making processes and tracking its progress, both of which support transparancy, as well as opportunities to influence it, such connection, sharing of information, a platform for developing ideas \cite{Afzalan2017}. %-{\color{orange}“The technical capacities of online participatory tools may facilitate specific types of decision-making processes by providing opportunities for deep dialogue, quick information sharing, or social mobilization. The type of technology may not directly influence the entire decision making process; but, it may promote certain types of processes by guiding how people can share or discuss ideas.}\cite{Afzalan2017}\\

A critical element of any spatial understanding, smart or participative, is the collection of data with a geospatial element. Beyond intermittent and reprentative polls of the community and implanted IOT devices capturing objective states of the environment, citizens themselves are a wealth of spatially routed information within a community\cite{Roche2012}. %- {\color{orange} “The active engagement of citizens in then a major requirement for smart operations of a city, and this needs for citizens to be spatially (digitally) enable. Capturing the sense of places in then a major stake for urban communities that look for a smarter way of development.”} \cite{Roche2012}\\ %- {\color{orange}“Smart urban solutions have to be built on the vision of citizens as active sensors on one hand and on spatial enablement of citizens via social networks on the other hand. These solutions have to be inline with improvement of navigation related spatial skills using geographical information and techniques for annotating spaces with digital information. These kind of solutions have also to be built on the potentials offered both by embedded sensors to crowdsource the process of collecting geo-referenced information about places in the city and social networks to disseminate this information and democratize access to it.”} \cite{Roche2012}\\ %- {\color{orange}“volunteered Geographical Information -- VGI - becomes the most prolific sources of information to characterize places.”} \cite{Roche2012}\\
Whether active or passive -- describing whether the data generation is consciously initiated by a participant (such as participating in a forum) or collected in the backgroud of regular activity (such as via a smart phone app), both of which should be intentionally shared if accessed by a third party -- and whether primarily focused towards citizen engagement (such as answering a poll) or extracted for such use (such as sentiment extraction of public social media posts), volunteered geographic information (VGI) is critical to the understanding of "citizens' social synergies in the urban context"\cite{Acedo2019, Evans-Cowley2010}. %-{\color{orange} “the role of the geographical perspective in taking another step forward to better understand citizens’ social synergies in the urban context.”} \cite{Acedo2019}\\ %-{\color{orange} Digital tools can increase public engagement in politics and urban planning without citizens even actively intending to contribute.} \cite{Evans-Cowley2010}\\
Especially in issues of public planning, the understanding of individuals' spatial context can realign the lens, and therefore the results, of community initiatives towards the people of whom it is composed. %-{\color{orange}“the understanding of the spatial relationship between SoP, SC and CE establishes novel spatial scenes based on human-city interactions. There is a potential for understanding the spatial relationships of social concepts (i.e., SoP, SC and CE) to provide alternate or completely new units of analysis for citizen participation. The definition and mapping of these individuals’ spatialities allow a more citizen-focused representation of the city.”} \cite{Acedo2019}\\
Though there continues to be a disparity between the understanding of places and the people who inhabit them\cite{Acedo2019}%-{\color{orange} “There is a shortage of empirical research on the interactions between people and places.” Justification of project} \cite{Acedo2019}\\} 
, these tools can establish a better connection between the \textit{where} and the \textit{why} and \textit{how} of spatial phenomenon and the perspectives of those who experience them\cite{Painho2013}.%-{\color{orange}“it is assumed that the use of techniques and tools derived from the Neo-Geography, in particular VGI, helps to develop an understanding of the complexity of geographical space and place, through the analysis of geographical phenomena and that this change may influence the attitudes of citizens to the discussion of the problem facing society.”}\cite{Painho2013}\\

%Implementation
%-{\color{orange}“Citizens’ characteristics and skills, and their attitudes towards using technology for participation should be also considered in choosing online participatory tools. Communities that will be directly or indirectly affected by plans are indispensable components of participatory processes.”}\cite{Afzalan2017}\\
%-{\color{orange}“Before using an OPT an organization must clearly define the goals for the tool’s use. Is the main goal of using the OPT to inform and educate citizens, follow up with citizens about particular aspects of the plan, engage citizens in a consensus-building process, resolve tensions between conflicting ideas, or build trust in communities? Is the goal of using OPTs to attract those who usually do not or cannot attend public meetings, or is the purpose of using OPTs to encourage excitement about a project in a community?”}\cite{Afzalan2017}\\
%-{\color{orange}“Community norms can also influence the organizations’ decision towards using new technologies. Cities’ efforts towards creating open data portals or developing transparency advisory boards are example of such activities.” }\cite{Afzalan2017}\\
%-Examples: MySideWalk, SeeClickFix, nextdoor, citysourced, placespeak, mindmixer \cite{Afzalan2017}
%-{\color{orange}“Minimizing cognitive complexity and enhancing ease of the use for hte PPGIS participant appears essential to increasing participation rates. And yet, there may be a contrary relationship between the simplicity of the PPGIS data capture methods and the resulting quality of the spatial data.”\cite{Brown2012}}\\
%-{\color{orange}“Perhaps the one constant factor that needs to be kept in mind as these choices are made is that technology will be alienating without participation.”\cite{Williams2016}}\\
%-{\color{orange}“loca-l governments are increasingly designing their websites from the standpoint of “what does the citizen need to know’ versus ‘what information do we have to provide,’ offering the potential for better engagement. However, very few municipal websites facilitate online public dialogue or consultation.”}\cite{Evans-Cowley2010}\\
%-{\color{orange}“organizations’ attitudes towards managing and controlling the online environment can influence engagement by encouraging or discouraging particular behaviors. For example, responding to participant questions in a timely manner may result in a higher ongoing participation rate.”}\cite{Afzalan2017}\\

- {\color{orange}“But more important for a smart city is its capability to capture the sense of places. A city is not a machine, bur rather made by people local actions and feelings.” \cite{Oliveira2021}}.\\

-{\color{orange}Distinguishing characteristics of local knowledge: “It is based on experience. It is developed over time by people living in a given community, and is continuously developing. It is embedded in community practices, institutions, relationships, and rituals. It is held by individuals or communities. It is dynamic and changing. Based on these characteristics, we may anticipate that local knowledge is unique from place to place. Therefore, the gazetteer used for geo-referencing local newspaper articles should be place-specific.”\cite{Cai2016}}\\












