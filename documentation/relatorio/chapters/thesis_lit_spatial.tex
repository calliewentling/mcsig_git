\subsection{Spatial information}
"Location is involved with everything"\cite{Bhattacharya2018}. %-{\color{orange} “Location is involved with everything, hence a spatial system is vital for better urban information management and spatial data infrastructure (SDI) creation.” } \cite{Bhattacharya2018}\\
Location is the definition of a point in space relative to the earth's surface. It describes, along with thematic and possibly temporal attributes, phenomena, such as where events happen or where something is spatially situated\cite{Longely2005}. Location can be defined in a variety of ways, though perhas most often interpretted by humans in two dimensional coordinates associated with a particular spatial reference system (SRS). A common example is the often used longitude and latitude coordiantes of spatial reference system identifier (SRID) 4326, that is associated with a spheroidal cartographic reference surface of the World Geodetic System (WGS84), and famously serving as the reference coordinates of the global positioning service GPS\cite{dma91}. Such formal definitions of location allow the connection of disparate phenomena or datasets by providing a common framework within which they can be compared \cite{Bhattacharya2018}. %-{\color{orange}“Location plays a vital part by helping understand relations between datasets.”}\cite{Bhattacharya2018}\\

\subsubsection{Geographic information systems}
Though often relevant for decision making at the personal, organization, and regional levels, spatial data is underutilized in formal analysis\cite{Bhattacharya2018}. %-{\color{orange} “the spatial aspects of data is being largely underutilized in processing, although the importance of spatial decision-making is now widely accepted”.}\cite{Bhattacharya2018}\\ %-{\color{orange}“Almost every decision that an individual or organization makes has some geospatial component.”}\cite{Bhattacharya2018}\\

Geographic information systems (GIS) are computer programs that suppor the collection, sharing, processing and visualiation of geospatial data and its resulting information\cite{McQueenBaker2019}. %-{\color{orange}“Geographic information systems (GIS) are computerised systems that allow users to electronically tag geographic locations with various information. Maps can then be constructed for visual representation.” Definition of GIS and applications\cite{McQueenBaker2019}}\\ 
This georeferencing (associating data to a map) is fundamental for understanding where people, things, and events are, were, or may be\cite{Xing2015,Rajabifard2009}. %-{\color{orange} Georeferencing data content: “users to add locations or location-related data content onto a map”.}\cite{Xing2015}\\ %-{\color{orange}“in modern society, spatial information is an enabling technology or an infrastructure to facilitate decision making.”}\cite{Rajabifard2009}\\ 
The resulting geogrpahical datasets, in the forms of maps and features, provide an opportunity to orient collaborators, share experiences, and challenge presumptions of the users\cite{Jiang2020, McQueenBaker2019}.%-{\color{orange} “Geographical datasets include maps and feature-based datasets, such as cadastral maps, national and regional boundaries, and other GIS-type data and information.” As opposed to earth observation datatype (in-situ or remote sensing). \cite{Jiang2020}}\\ %-{\color{orange}“Evolution of, and increased access to, GIS technology provides an opportunity to visually create and share experiences, information, and perspectives across a broader audience, breaking geographic barriers for users through multimodal construction of meaning.”\cite{McQueenBaker2019}}\\ %-{\color{orange}An opportunity for maps to “contradict the dominant narrative” and “increase understanding”.\cite{McQueenBaker2019}}\\
One of its most powerful opportunities of this dynamic description is to address problems by anchoring relations between datasets and developing spatially considerate solutions to identified problems \cite{Bhattacharya2018, Rajabifard2009}. %-{\color{orange}“Location plays a vital part by helping understand relations between datasets.”}\cite{Bhattacharya2018}\\ -{\color{orange}“Spatial information can be a unifying medium in which linking solutions to location.”}\cite{Rajabifard2009}\\ %- {\color{orange}  “Location-based services and information are the basic components needed to dynamically describe and represent places’ life” } \cite{Roche2012}\\
It is no surprise, then, that location data is already considered valuable and increasingly being incorporated into at all scales of community operations\cite{Bhattacharya2018, Roche2012}. %- {\color{orange} “Location data is now commonly regarded as the fourth driver in the decision-making process.”} \cite{Roche2012}\\ %-{\color{orange} “Nowadays it is increasingly imperative to capture data at all spatial scales, local to global, and extract useful information from it ubiquitously and regularly.”}\cite{Bhattacharya2018}\\ %-{\color{orange}“the concept of geo-data democratization is gaining momentum and is being described as the next big disruption.”}\cite{Bhattacharya2018}\\ %- {\color{orange} “SDI and spatial technologies are now used routinely in decision making to support city planning and forecasting and at the same time to address some of the world’s most pressing societal problems.”}\\
One of its most valuable characteristics is the ability to display relational values across time and space\cite{McQueenBaker2019}. %-{\color{orange}“The spaces in which we live, work, and play affect our lives both positively and negatively, and GIS provides a way to visually convey relational meaning of spaces.'' \cite{McQueenBaker2019}}\\
The convolution of GIS technology, the availability of GPS infrastructure, and machine learning techniques promote a myriad of real-time and spatial services with research, commercial, and security implications \cite{Barns2020, Al-Olimat2018} % -{\color{orange} “Decades of technology R\&D, including, not least, US Government investment in relatively ‘open’ data made available via Global Positioning Satellite (GPS) infrastructure have unleashed the potentials of real-time location tracking and the myriad services this data can be used to enhance, via machine learning techniques.”\cite{Barns2020}}\\ %-{\color{orange}“In context-aware computing, location is a fundamental component that supports a wide-range of applications.”\cite{Al-Olimat2018}}\\


%\subsubsection{Location}
%-{\color{orange}“Locality is used to describe a more precise area.”\cite{Imani2019}}\\
%-{\color{orange}Hyperlocality: “a spatiality that is endemic - i.e., locationally specific - to the individual, real-time positionalities of digital platform users.” Ex:  “one of the most immediate examples of hyperlocality is the ‘blue dot’ on the mobile Google Maps interface, which visually centers the map data on the real-time location of the user.”\cite{Leszczynski2019}}\\
%
%\subsubsection{Place}
%- {\color{orange} “From a practical point of view, and in the smart city context, “spatial enablement” refers to the individuals’ (or collective) ability to use any geospatial information and local technology as a means to improve their spatiality, that is to say, the way they interact with space and other individuals on/in/through space. Spatiality is the dynamic component of place making.”}
%
%
%\subsubsection{Social}
%- {\color{orange} “We find that the sense of place is significant and positively correlated with social capital, while the latter also significantly explains civic engagement at the individual level.”} \cite{Acedo2019}\\
%-{\color{orange}“A ‘sense of place’, of rootedness, can provide - in this form and on this interpretation - stability and a source of unproblematical identify.”}\cite{MasseyD1991}\\
%-{\color{orange}Psychocraphpics: ``the prevailing interests of people in the area''}\cite{Chiappinelli2020}\\

%-{\color{orange}Globalization, “speeding up and spreading out” = Time-space compression. “Movement and the communication across space, to the geographical stretching-out of social relations, and to our experience of all of this.”}\cite{MasseyD1991}\\
%-{\color{orange}“What is it that determines our degrees of mobility, that influences the sense we have of space and place?”}\cite{MasseyD1991}\\
%-{\color{orange}“The current speed-up may be strongly determined by economic forces, but it is not the economy alone which determines our experience of space and place.”}\cite{MasseyD1991}\\
%-{\color{orange}“power geometry of time-space compression. For different social groups, and different individuals, are placed in very distinct ways in relation to these flows and interconnections… DIfferent social groups have distinct relationships to this anyway differentiated mobility: some people are more in charge of it than others; some initiate flows and movement, others don’t; some are more on the receiving end of it than others; some are effectively imprisoned by it.”}\cite{MasseyD1991}\\
%-{\color{orange}“Differential mobility can weaken the leverage of the already weak. The time-space-compression fo some groups can undermine the power of others.”}\cite{MasseyD1991}\\

%-{\color{orange}“We need, therefore, to think through what might be an adequately progressive sense of place, on which would fit in with the current global-local times and the feelings and relations they give rise to, and whcih would be useful in what rae, after all, political struggles often inevitably based on place. The question is how to hold on to that notion of geographical difference, of uniqueness, even of rootedness if people want that, without it being reactionary.”}\cite{MasseyD1991}\\
%-{\color{orange}“the potential to build understanding of spatial context as both socially and politically constructed, but also the tendency to see a map as bounded data in finality.”\cite{McQueenBaker2019}}\\
%-{\color{orange}“Place is constructed meaning within [physical spaces]. These places of meaning are a crossing of ‘discursive, interpretive, livesd and imagined practices. Spaces are made up of places with socially created, accrued meaning. Places support the formation of identity, security, and belonging within spaces. Connecting with ideas of new materialism, spatial researchers see this formation as fluid rather than fixed… Humans impact and are impacted by the social relations in those spaces.”\cite{McQueenBaker2019}}\\
%-{\color{orange}“Looking at the politics of spaces helps policy makers, community members, educators, and students understand oppressive cycles of policies imposed on communities.”\cite{McQueenBaker2019}}\\
%-{\color{orange}“spatial theory offers a glimpse into the complicated relationships between humans and their surrounding contexts. GIS could be first, a snapshot of those relationships and second, a conversation starter for further investigation and action.”\cite{McQueenBaker2019}}\\
%-{\color{orange}New materialism: “the material world is not lifeless soil but something more dynamic.”\cite{McQueenBaker2019}}\\
%-{\color{orange}Maps “are objects of analysis that are linguistically and materially [hypercontexually] entangled… resulting in a continual state of becoming. In other words, the material does not construct our reality any more than the discursive defines us socially -- instead, it is how the relationship morphs, depending on the context.”\cite{McQueenBaker2019}}\\
%-{\color{orange}“When objects, words, people, or places come together, they make something altogether new.” onto-epistem-ology: “the ability to use maps as meaning-making of marked experiences in time an dspace, with the understanding that this is not a static representation of meaning but a point in an evolving state.”\cite{McQueenBaker2019}}\\
%-{\color{orange}GIS is a tool to powerfully represent data and effect change by demonstrating how the spatial affects people, and new materialism forces us to concede GIS visual data as already obsolete. The relational ontologies of new materialism disrupt the static representation of a fixed point on a map, bringing the geography and biology of social life into conversation ‘in a discipline which for a long time intentionally decided to ignore the issues arising from this connection.’ “\cite{McQueenBaker2019}}\\

\subsubsection{Feature definition}
Where something occurs is stored in raster (pixelated image) or vector (feature) format. Features can range from simple geometries, such as points, lines, or polygons (0-, 1-, and 2-dimensional features, respectively), or develop into more complicated multi-element collections, mixed element collections, or three dimensional definitions (incoporating a z-axis in addition to x- and y-). The type of feature to be used for a particular application depends on the information one wants to convey. For example, points are presumed to have an unknown range of influence, whereas polygons impose a boundary on whatever they represent \cite{Brown2012}. This can be problematic when attempting to describe a continuous value, or a feature with a fuzzy or inconsistent boundary. %-{\color{orange}“In mapping a PPGIS attribute with a point, the spatial attribute of interest is presumed to extend outward from the point in some unknown distance in some unknown direction. For PPGIS attributes identified as polygons, the participant is required to create boundaries that necessarily bifurcate the PPGIS attribute on the landscape, many of which are best viewed as continuous. For polygons, one can argue that the inaccuracy of a single point to represent the spatial attribute is replaced by the inaccuracy of an infinite number of points along the polygon’s edge. Alternatively, one can argue that some areal boundaries are better than none and increase the accuracy, even if indeterminate, of the attribute being identified.”\cite{Brown2012}}\\ -{\color{orange}“The nature of the PPGIS attribute being identified, in particular, affects accuracy.” What does a point representation of a line or area or other fluid definition mean? “the level of accuracy may be indeterminant”. How hard are polygon boundaries? \cite{Brown2012}}\\
The choice also depends on the size of the dataset, as improved performance is noted in polygon representation of information of sparse sampling due to the high data density required for point analysis \cite{Acedo2019}. %\color{orange} Polygons (vs. points): ease of implementation, “better encompass of a high range of spatial scales, (from an armchair to the whole earth)”, “better performance of polygon features when there is a limited spatial dataset.”} \cite{Acedo2019}\\% {\color{orange} “the same PPGIS attributes identified by points and polygons will converge on a collective spatial ‘truth’ within the stuy area provided there are enough observations, however, the degree of spatial convergence varies by PPGIS attribute type and the quantity of data collected.”\cite{Brown2012}}\\% -{\color{orange}“The use of points for mapping PPGIS attributes and aggregating areas through density mapping constitutes a conservative approach to spatial inference about place significance, but the data demands for point collection are considerably higher than for polygon features.”\cite{Brown2012}}\\
The choice of feature type affect both the analytical processing as well as the deducations made when viewing the results, the latter being further affectedb y placement and marker representation \cite{Brown2012}. %-{\color{orange}“the spatial feature chosen for soliciting spatial information may influence both the empirical results and the inferences that can be made.”\cite{Brown2012}}\\ %-{\color{orange}“Precision is a measure of the exactness in placing the PPGIS marker on the map… The precision of market placement on the map depends on a number of variables including marker size and map scale as well as participant characteristics such as visual acuity and physical dexterity. Flexible mapping environments that provide multiple map scales and marker sizes such as Google Maps can, in theory, enhance the precision of marker placement.” \cite{Brown2012}}\\
Ultimately, the application and requirements of the system will determine the selection of feature representation. Considering needs of accuracy versus precision, data literacy of the readers, opportunities for clarification from additional materials, and purpose will assist in the selection of vector type. %-{\color{orange}“For PPGIS attributes used in regional planning applications conducted at a larger scale, concern with mapping precision is relatively small compared to the accuracy of the geographic area represented by the marker. Accuracy reflects how well the marker approaches the true spatial dimensions of the attribute being mapped.  Accuracy in PPGIS is influenced by a number of variables including the nature of the PPGIS attribute being mapped (i.e., clarity in operational definitions and instructions enhance accuracy), the quality of the mapping environment (e.g., what base map features are included), and respondent characteristics such as map literacy. ”\cite{Brown2012}}\\

%%%%%%%%%%%%%

A place, by contrast to a location, is a geospatial definition, and potentially much more. They include the objective, such as descriptions of the objects physically present, such as a river or a building, but there are other elements to consider. Places are subject to the multiple identies strewn upon them by those who experience them under different conditions, formed by direct engagement or passed anecdotes\cite{MasseyD1991}. %-{\color{orange}Places don’t have “single, essential identities”, nor are they based on “internalized origins”.}\cite{MasseyD1991}\\ %-{\color{orange}“While [a place] may have a character of its own, it is absolutely not a seamless, coherent identity, a single sense of place which everyone shares. It could hardly be less so. People’s routes through the place, their favourite haunts within it, the connections they make (physically, or by phone or post, or in memory and imagination) between here an the rest of the world vary enormously. If it is now recognised that people have multiple identities than the same point can be made in relation to places. Moreover, such multiple identities can either be a source of richness or a source of conflict, or both.”}\cite{MasseyD1991}\\ %-{\color{orange}“How, in the face of all this movement and intermixing, can we retain any sense of a local place and its particularity? An (idealised) notion of an era when places were (supposedly) inhabited by coherent and homogeneous communities is set against the current fragmentation and disruption.”}\cite{MasseyD1991}\\
These dynamic and overlapping definitions mean that places can change over time and space. The same location can refer to multiple places, depending on the context of the definer. %-{\color{orange}“On the one hand communities can exist without being in the same place - from networks of friends with like interests, to major religious, ethnic or political communities. On the other hand, the instances of places housing single ‘communities’ in teh sense of coherent social groups are probably - and, I would argue, have for long been - quite rare. Moreover, even where they do exist this in no way implies a single sense of place. FOr people occupy different positions within any community.”}\cite{MasseyD1991}\\ %-{\color{orange}“what gives a place its specificity is not some long internalised history but the fact that it is constructed out of a particular constellation of social relations, meeting and weaving together at a particular locus.”}\cite{MasseyD1991}\\ %-{\color{orange}“If places can be conceptualised in terms of the social interactions which they tie together, then it is also the case that these interactions themselves are not motionless things, frozen in time. They are processes.”}\cite{MasseyD1991}\\ %-{\color{orange}“places do not have single, unique ‘identities’; they are full of internal conflicts.”}\cite{MasseyD1991}\\
Place, then, can have history or nostalgia, and these informal relationships may be constantly morphing\cite{Roche2012}. %-{\color{orange}“Place is often uses in the sense of community or neighbourhood, implying an informal relationship to an area surrounding the individual’s place of residence” (refers to Goodchild [8])} \cite{Roche2012}\\%-{\color{orange}“The specificity of place is continually reproduced but it is not a specificity which results from some long, internalised history. THere are a number of sources of this specificity - the uniqueness of place. There is the fact that the wider social relations in which places are set are themselves geographically differentiated. Globalisation (in the economy, or in culture, or in anything else) does not entail simply homogenisation. On the contrary, the globalisation of social relations is yet another source of (the reproduction of) geographical uneven development, and thus of the uniqueness of place.”}\cite{MasseyD1991}\\ %-{\color{orange}“There is specificity of place which derives from the fact that each place is the focus of a distinct mixture of wider and more local social relations. There is the fact that this very mixture together in one place may produce effects which would not have happened otherwise. And finally, all these relations interact with and take a further element of specificity from the accumulated history of a place, with that history itself imagined as the product of layer upon layer of different sets of linkages, both locand and to the wider world.”}\cite{MasseyD1991}\\ %of place would recognize that, without being threatened by it. What we need… is a global sense of the local, a global sense of place.”}\cite{MasseyD1991}\\
This is especially apparent when considering a microcosm such as a neighborhood, community, area as a place. The experience of those who live, work, and visit a place and constantly being fedback into these areas, leveraging local knowledge and expectations of what the place is\cite{Cai2016}. %-{\color{orange}Distinguishing characteristics of local knowledge: “It is based on experience. It is developed over time by people living in a given community, and is continuously developing. It is embedded in community practices, institutions, relationships, and rituals. It is held by individuals or communities. It is dynamic and changing. Based on these characteristics, we may anticipate that local knowledge is unique from place to place. Therefore, the gazetteer used for geo-referencing local newspaper articles should be place-specific.”\cite{Cai2016}}\\
For all of these reasons, it can be challenging to associate thematic attributes attempting to characterize a place as it pertains to predefined boundaries, as the personal perception of these areas, even the locations of their borders, may differ from the official records \cite{Acedo2019}. %-{\color{orange} “the citizens’ perception of pre-established administrative boundaries can differ from the “real” one nd, consequently, whole administrative boundaries might not cover the SoP, SC, and aCE of all its dwellers.”} \cite{Acedo2019}\\

Cross-border mapping is a specific instance in which boundaries can be a challenge. Boundaries seem to instill a binary adherence, an inclusion or exclusion that, when applied to populations, can be analytically and socially divisive \cite{Acedo2019, MasseyD1991}. %-{\color{orange} Massey 1991: “highlights the problem of recurrently trying to draw boundaries to the conception of place and place-related concepts that, in inherently, distinguishes between an inside (e.g., us) and an outside (e.g., them). She also supports that there is no need to conceptualize boundaries in order to define place, advocating that place is a process of social interactions.”} \cite{Acedo2019}\\ %-{\color{orange}“A particular problem with this conception of place is that it seems to require the drawing of boundaries. Geographers have long been exercised by the problem of defining regions, and this question of ‘definition’ has almost always been reduced to the issue of drawing lines around  a place.”}\cite{MasseyD1991}\\ %-{\color{orange}“cross-border mapping can contribute to cross-border cooperation.”\cite{Witschas2004}}\\
More likely, the data on either side of the line are heterogeneous, more fluidly changing state over a shore of values, rather than a counterposition the other on the opposite side of the enclosure %\cite{Witschas2004}. %-{\color{orange} “The crucial point of transboundary spatial analysis is data heterogeneity.”\cite{Witschas2004}}\\
In fact, political borders tend to be natural points of exchange of people, things, and ideas in a way that are least adherent to the administrative areas to which they belong \cite{Xing2015}.%-{\color{orange} “Borderland is a natural transition and convergence area where people, goods, services, and ideas flow across boundaries or seas from state to state. Such cross-border commonalities, which cannot be divided by politically dictated and artificial boundary lines, potentially contribute to sustainable development within the world.” Definition of borderland to describe areas that don’t convene to administrative boundaries. “A borderland region generally refers to the land area adjoining and outside state boundary lines, or hte ocean area among maritime neighbors, and also has different characteristics or geographic conditions than the inner or central parts of the neighboring nations.”}\cite{Xing2015}\\ %
Therefore, any static representation is already a distortion of reality (just as or more than any data representation is an abstraction) \cite{McQueenBaker2019} %-{\color{orange}“Any static method used to represent reality, whether linguistically, discursively, or culturally construed, distorts reality in some way. Researchers acknowledge the material experiences of participants while anticipating that these realities flow and change.”\cite{McQueenBaker2019}}\\


%Massey even goes so far as to suggest that we can reimagine place definitions as momentary states in time, or inextricabley linked to the elements external to themselves \cite{MasseyD1991}. %-{\color{orange}“Instead then, of thinking of places as areas with boundaries around, they can be imagined as articulated moments in networks of social relations and understandings, but where a large portion of those relations, experiences and understandings are constructed on a far larger scale than what we happen to define for that moment as the place itself, whether that be a street or a region or even a continent. And this in turn allows a sense of place which is extroverted, which includes a consciousness of its links with the wider world, which integrates in a positive way the global and the local.”}\cite{MasseyD1991}\\ %-{\color{orange}“places do not have to have boundaries in the sense of ivisions which frame simple enclosures… Definition in this sense does not have to be through simple counterposition to the outside; it can come, in part, precisely through the particularity of linkage to that ‘outside’ which is therefore itself part of what constitutes the place.”}\cite{MasseyD1991}\\

One tool to address these fluid definitions is neogeography, a new way of understanding place by combining location based services (LBS) and volunteer geographic information (VGI) \cite{Painho2013}. %-{\color{orange}NeoGeogrpahy: Location based services (LBS) and volunteer geographic information (VGI) “This change of paradigm of functioning of the Internet, combined with strong social networks diffusion, changed the concept of community.”} \cite{Painho2013}\\
As our movements have become greater in displacement and in number, so too have our connections to the people and places we encounter along the way. In that sense, communities are expanding beyond the confines of walkability and extending around the globe. Many of the same tools that allow us to remotely connect with each other include geolocation, making these social media users producers of vast quanitites of spatial data. This can be used to identify new, digitally relevant points of interest of areas where people convene, or be leveraged as expert data when users describe their own communities \cite{Roche2012}. %- {\color{orange}  “Indeed with the exponential growing of location-based social networks (geosocial), Geoweb 2.0 and geoinformation crowdsourcing, citizens are increasingly involved in the the production of geographic information. This kind of information, voluntarily produced and diffused by people, mainly refers to the places they live/use. Indeed, people whom live in a place are often the “experts” of this place.”} \cite{Roche2012}\\ %- {\color{orange} “Spontaneous and localized contributions of individuals (localized tweets, Facebook Places or foursquare chek-in) are most often materialized by Points of Interest (POI). This POI could be considered as new forms fo spatial projections of social relationships and human spatiality.”} \cite{Roche2012}\\
They are also, quite convenently, composed of the building blocks of geographic data\cite{Longely2005}. %{\color{orange}"Geographic data are built up form atomic elements, or facts about teh geographic world. At its most primitive, an atomof geographic data (strictly, a datum) links a place, often a time and some decriptive property."\cite{Longely2005}}



\subsubsection{GeoIntelligence}
%%-{\color{orange}Disaster response: “66\% of enterprises rank Location INtelligence as either critical or very important”\cite{Hintz2020}}\\
%%-{\color{orange}Deriving and delivering intelligence: extraction/collection, integration/fusion, iltering/cleaning, enrichment/analysis, distribution/consumable\cite{Hintz2020}}\\
%%-{\color{orange}“Geospatial intelligence, or the frequently used term GEOINT, refers to the discipline of exploitation and analysis of satellite imagery and other forms of earth observation data to describe, assess, and visually depict physical features and geographically referenced activities on the earth. The term GEOINT is typically used for the defense and internal security domains and offers the capability of monitoring, predicting and countering threats, while helping strategize and support various field operations.” \cite{Datta2018}}\\
%%-{\color{purple}Intentionally georeferenced data can contribute to GEOINT as a geolocated data source in addition to other imageries and info fontes.\cite{Datta2018}}\\
%%-{\color{orange} Trends: AI and automation, social media and mobile data, analytics-as-a-service, drive for cloud, short shelf life of technologies, geospatial information science\cite{Datta2018}}\\
%%-{\color{orange} “Smartphones in every hand, increasing use of social media, and sensors in transport vehicles have turned human beings into sensors. And this is what is driving the next wave of strategy and innovation for the GEOINT community. With the convergence of new sensors and social analytics, it is possible for defence organizations to access multi-source data for enhanced decision making. As more terrorist/criminal organizations take to social networks like Twitter and Facebook to communicate their efforts, we have seen defence and intelligence organizations are using the right data analytic tools for mining this data to gain a better operational picture of enemy activity.”\cite{Datta2018}}\\
%%-{\color{orange}Geospatial Information Science: “Different people behave in different ways and culture and location plays an important role in it. With terrorist/criminal activities rising, it is becoming paramount for defense and intelligence communities to learn how humans impact their environments and vice versa; how often cultural norms differ in relation to various environments. This will also enabler he GEOINT community to effectively use technology to capture, manage and analyze geographic information that supports leaders in making informed decisions in complex environments.”\cite{Datta2018}}\\
%%-{\color{orange} “Who? What? Why? When? Where? These questions form the basis of human exploration. They are fundamental to knowing and understanding our world. Their answers are essential to information gatherings, storytelling,and problem solving.” NGA SHOW THE WAY video\cite{nga}}\\
%%-{\color{orange} “Everything on Earth, from its watery depths to its hughes peaks,  can be measured in space and time, whats known as Geospatial Intelligence or “GEOINT.”\cite{nga}}\\
%%-{\color{orange} Situational awareness. “Tell me everything that is happening” prior to forming specific questions \cite{Varanda2020}}\\
%%-{\color{orange} “Hot spot analysis refers to the use of geostatistics and time series analysis to find out the most significant spatial hot spots of a certain indicator and their respective temporal trends.”\cite{Varanda2020}}\\
%%-{\color{orange}ORACULO peacekeeping leverages remote sensing imagery (IMINT), relationship development (HUMINT), reconnaissance, and open sources (OSINT). OSINT: “Thus, though every other way of collecting events promises more reliable results, if we want fast, better than nothing results, we have to turn to open sources, such as news websites.”\cite{Varanda2020}}\\
%%-{\color{orange} “To whom does the information flow? The Media, of course, which has continued to operate throughout the conflict.”\cite{Varanda2020}}\\
%%-{\color{orange}“Extract events from open sources / manage event geodatabases / mine event space time patterns / improving situational awareness for a given context.”\cite{Varanda2020}}\\
%%-{\color{orange} Merge multiple reportings of the same event based on temporal, spatial, and thematic proximity\cite{Varanda2020}}\\
%%-{\color{orange} “Without location context, first responders are unable to decide where or how to respond to information they receive. This is especially true during natural disasters when geographic content is necessary for dispatching appropriate emergency response.”\cite{Snyder2019}}\\
%%-{\color{orange}“Understanding that meaning shifts through an interplay of social, linguistic, and material interactions, GIS has the unique ability to show fluidity yet connectedness across geographic borders, providing intriguing opportunities as a research tool.”\cite{McQueenBaker2019}}\\
%%
%%-{\color{orange}“Though GIS studies have the potential to represent a fluidity through constant access to technology that written reports often lack, they are still bounded by restrictions of language and symbols, as well as technical and financial resources.”\cite{McQueenBaker2019}}\\
%
\subsubsection{Geoportals/mashups}
%-{\color{orange}“There is a big need for spatially referenced data creation, analysis and management.”}\cite{Bhattacharya2018}\\
%-{\color{orange} “The launch of Google Earth in 2005, and the availability of Google maps programming interface (API), as well as other initiatives, have transformed the way Internet users relate to geographic information (GI). The transfer of the information-creation processes from the specialized domain of Geographic Information Science to the field of action of the non-experts, the fact that other citizens, in addition to the geographers, cartographers or GI specialists, can create their maps with their own content, is to radically change the domains of interest and application of these mechanisms and to impact the criteria for the collection, analysis, implementation and the standard of truth of the information, with implications for information access, participation, power balance and nature of the data.”} \cite{Painho2013}\\
%-{\color{orange} GIS: “the opportunity to  open spatial information to all stakeholders (presumably leading to better policy-making) and the idea that spatial analysis and outputs (.e. maps) can persuasively convey ideas.”}\cite{Kleinhans2015}\\
%-{\color{orange} “innovative and imaginative geo-visualization interfaces such as Google Maps or Open Street Map - made possible by Web 2.0 technologies -- have created low-key opportunities for almost any citizen with an Internet connection to generate and publicize their own maps and geographic information.”} \cite{Kleinhans2015}\\
%-{\color{orange}“Organizations have different capacities to use new participatory methods. For example, private organizations or planning consultants may be better equipped than local governments to launch an online tool or collect and analyze the data.”}\cite{Afzalan2017}\\
%-{\color{orange}“Internal and external organizational collaborations influence the successful application of new technologies and methods within institutional systems. Collaboration with outside organizations can facilitate the incorporation of new technologies and data sources.”}\cite{Afzalan2017}\\
%-{\color{orange}“There has been a need to develop automated integral spatial system to sense and categorize events and issue information that reaches users directly.”}\\
%-{\color{orange}“It has been proven that the spatial analysis of data gives more meaning to the information extraction and hence enables easier assimilation of large volumes of data. Presently the systems that implement such processes are limited in effect by not utilizing all the data due to their standalone nature, offline or disconnected design, lack of spatial capabilities, unintegrated approach, and temporally disjoint.”}\cite{Bhattacharya2018}\\
%-{\color{orange} “The solution could be addressed through integrating: data source, spatial data platform, data understanding, knowledge base, inferencing an dvisuzalition into single, well-connected online real-time system.  Such a spatial expert system (ES) with knowledge base (KB) will not only serve the critical research of spatializing developmental works but do so to any research relying on real time data capture and analysis with spatial domain of data being the unique enabler.”}\cite{Bhattacharya2018}\\
%-{\color{orange}“A new paradigm of more open, user-friendly data access is need to ensure that society can reduce vulnerability to spatial data variability and change, while at the same time exploiting opportunities that will occur.”}\cite{Bhattacharya2018}\\
%-{\color{orange}“Although research scientists have been the main users of these data, an increasing number of resource managers need and are seeking access to spatial data to inform their decisions, just as a growing range of policy-makers rely on spatial data to develop spatial change strategies. With this gravity comes the responsibility to curate spatial data and share it more freely, usefully, and readily than ever before.”}\cite{Bhattacharya2018}\\%!!!
%-{\color{orange}“The interactive graphical user interface (GUI) allows for data visualization manipulation and sharing.”}\cite{Bhattacharya2018}\\
%-{\color{orange}THe ability to harness and render this information in a location context is a major challenge.”}\cite{Bhattacharya2018}\\
%-{\color{orange}“GeoWiki, a geographical semantic wiki system that was introduced in [10], can parse and store multi-source geographical knowledge, mash up with Google Maps and serve geospatial decision-making.”}\cite{Xing2015}\\
%-{\color{orange}“A mashup is a Web application that aggregates multiple services to achieve a new purpose. It is an important feature of Web 2.0 and can be used with software provided as a service. With various mashup techniques, it is convenient for developers to obtain data from a variety of data sources on the web and to integrate these data to build new applications.” }\cite{Xing2015}\\
%-{\color{orange} “A map mashup therefore combines at least one map data source or services with added information, often geo-referenced to the map data, to create a new map.”}\cite{Xing2015}\\
%-{\color{orange} “Geoportals are a consolidated web-based solution to provide open spatial data sharing and online geo-information management.”}\cite{Jiang2020}\\
%-{\color{orange} “geoportals usually provide access to distributed data systems, offering maps, data discovery, and data downloads. Some of them are also capable of offering online analysis and processing service, enhanced semantic search engines, and dynamic visualization tools.'' \cite{Jiang2020}}\\
%-{\color{orange}There is an ongoing challenge to effectively manage and communicate the vast and growing amounts of spatial data and geo-information. \cite{Jiang2020}}\\
%-{\color{orange} “oVer the past few decades, the concept of geoportals has emerged as one of the key solutions for spatial data and geo-information accessing and sharing.” most recently through the internet. \cite{Jiang2020}}\\
%-{\color{orange}Geoportal: “a point of access to spatial data and geo-information. It is able to provide a geospatial data inventory linking to an inclusive collection of spatial data, geographic information, online services, and data processing tools. In this article, the use of the term ‘geoprotal’ refers to the human-to-machine interface performing as a single point-of-access to spatial data and geo-information systems, offering sharing capabilities and connecting between geospatial data providers and end users. It is typically employed as a web-based graphical user interface (GUI) equipped with functionalities for accessing Earth observation data and geographical information.”\cite{Jiang2020}}\\
%%-{\color{orange} Geoportal as a term from 2005: “refers to a web environment that acts as a gateway to connect with a Spatial Data Infrastructure (SDI).\cite{Jiang2020}}\\
%%-{\color{orange} SDI: “one of the key elements of a regional or global SDI is the capability of searching for viewing, transferring, ordering, advertising, and disseminating spatial data from numerous sources from the Internet.” \cite{Jiang2020}}\\
%%-{\color{orange} Geoportal vs. Spatial web portal (synonym) \cite{Jiang2020}}\\
%-{\color{orange}Drivers of geoportal advances: “(i) scientific geospatial projects and applications, (ii) international organizations, (ii) governmental agencies, and (iv) commercial purposes”.\cite{Jiang2020}}\\
%- {\color{orange} International organizations: Sharing data from “heterogeneous data sources”. \cite{Jiang2020}}\\
%-{\color{orange}Governmental drivers: open government policies. “These portals, which were built by answering the open government call (or named the e-government), act as the gateways, anchors, or major starting sites for governmental data, no matter whether the data is spatial or non-spatial in nature.” “Openness and transparency are fundamental to ensuring citizens’ trust in their governments. Thus the objective of a government geoportal is to foster greater transparency and accountability, providing information available to the public from digital technologies.” Example of Geospatial OneStop (GOS) in the USA.\cite{Jiang2020}}\\
%-{\color{orange} Commercial geoportal: connect users to digital resources, including geospatial datasets.\cite{Jiang2020}}\\
%-{\color{orange}“Data is the content that geoportals provide to end users. A single geoportal may provide heterogeneous geospatial datasets coming from multiple data sources.”  Geographical and earth observation dataset types, additional products, models, and aggregation, data harmonization\cite{Jiang2020}}\\
%-{\color{orange} “The geoportals can facilitate researchers, government officers, and ordinary users in helping them to find the data they needed, with basic searching services equipped in the geoportal.”\cite{Jiang2020}}\\
%-{\color{orange} “a geoportal is normally a general-purpose tool to discover spatial data and geo-information, but often there are communty-specific needs that require customized geoportals with dedicated tools for scientists, policy-makers, and the general public.” \cite{Jiang2020}}\\%Validation of tool development on a community level
%-{\color{orange}“GEoportals should also play a role as the interface of so-called social spatial data infrastructure. New system architectures (e.g. Linked data and Semantic Web) to establish a shared information space relying on URLs and Resource Description Framework (RDF) can also provide another solution for accessing geoportal data.” \cite{Jiang2020}}\\
%%-{\color{orange}“GEoportals should also play a role as the interface of so-called social spatial data infrastructure. New system architectures (e.g. Linked data and Semantic Web) to establish a shared information space relying on URLs and Resource Description Framework (RDF) can also provide another solution for accessing geoportal data. \cite{Jang2020}}\\
%-{\color{orange}“Geoportal is considered as the entry point and the human-to-machine interface of the data and information management system.”\cite{Jiang2020}}\\
%-{\color{orange}“Promote a common way of accessing data for an enterprise and any communities that need to make use of that data”\cite{Hintz2020}}\\
%
\subsubsection{Commercial geospatial platforms}
%-{\color{orange}The definition of the search criteria by the user is usually done based on a form, defining the spatial, temporal and thematic coverage of the searched data. In the context of geodata search the use of maps on which searched areas can be defined have become standard – here the region of interest is selected by identifying a point, a rectangle, a polygon or some pre-defined region (e.g. a country). In most cases alternatively geographical names or coordinates can be searched. Especially the latter approach is less intuitive, but also the use of geonames requires some care as very often the way in which names are written differs depending on the language and special national character sets are used.\cite{Eisl2020}}\\
%-{\color{orange}Platform ecosystem: “technologies, platforms, data points, user interfaces, relational infrastructure and application programming interfaces (APIs), service providers, investor-financiers and customers”\cite{Barns2020}}\\
%-{\color{orange} “‘the platform’ is not only a particular computational structure but also a discursive innovation, which works on behalf of its business owners to recast conditions of value-sharing between companies and their marketplace as ostensibly ‘open’. These ecosystem-relationships make possible conditions of algorithmic control, subtly shaping, nuding and ‘steering’ urban behaviors from infinitesimal to global scales, in ways that point to new asymmetries of urban governance in a world of connected devices, things, people and infrastructures.”\cite{Barns2020}}\\
%-{\color{orange} GPS enabled smart phones enables the superior experience of ridesharing to traditional taxis.\cite{Barns2020}}\\
%-{\color{orange} “No local knowledge is required: the app provides the requisite information to fulfill the transaction”.\cite{Barns2020}}\\
%-{\color{orange} “the value harvested by platforms is in the data generated through digitally mediated transactions and relationships, which informs the terms dictated by platform companies for wider data-driven applications.”\cite{Barns2020}}\\
%-{\color{orange} “The computational architecture put in place across a platform ecosystem hinges fundamentally on the role of the Application Programing Interface, or API, which enables different kind of software to connect.”\cite{Barns2020}}\\
%-{\color{orange}“Platform architecture can thus be read as highly recombinatory, ingesting and harvesting data from a highly porous, intermediated platform ecosystem of multisided relationalities and transactions, while underpinned by the protocols and modes of value-exploitation that are closely controlled by platform owners.”\cite{Barns2020}}\\
%-{\color{orange}“The tactics adopted by platforms firms to entice users to choose one service over another are described as ‘steering tactics”. For economists, steering tactics are central to the operations of successful platforms, which devote significant attention towards and investment in strategies that can enhance the overall value of the platform to its users.”\cite{Barns2020}}\\
%-{\color{orange}“geolocation organized five ‘platform affects’ for digital platform ecosystems. (i) affective space-times of hyperlocality and real-timeness; (ii) experiential affects of smoothness; (iii) affects of connectedness; (iv) affects of trust; and (v) affective value.“\cite{Leszczynski2019}}\\
%-{\color{orange}“THe merging of digital location with web content, services, and interfaces is pervasive to the extent that digital location - or simply ‘geolocation’ - now features as a native design, logistical, and organizational component of digital platform interfaces, utilities, and affordances.”\cite{Leszczynski2019}}\\
%-{\color{orange}Social media updates (geotag), ridehailing (geologistics), dating (search radius definition), accommodation (interactive map-based filtering)\cite{Leszczynski2019}}\\
%-{\color{orange}“Initial world by geographers has begun to preliminarily attend to the factors atively underwriting the integration of geolocation with digital economies. Yes this work has been largely market-oriented, engaging geolocation as first and foremost an economic commodity.”\cite{Leszczynski2019}}\\
%-{\color{orange}Platform affect: “informs an understanding of digital platform ecosystems as affectively oriented and as expressing a capacity for orienting affects. Platforms generate, accumulate, and cycle affects within ecosystems; they also draw on the affective capacities of other techno-material formations, such as geolocation, to organize affects for digital ecosystems.”\cite{Leszczynski2019}}\\
%-{\color{orange}“as geolocation is put into circulation, it expresses a capacity to affect - perturb, animate, align, mobilize, organize, dis/assemble - other things, both humans and other-than-human alike. Affect is useful for grappling with what geolocation ‘does’, how it ‘does’ what it does, and to what ends.”\cite{Leszczynski2019}}\\
%%-{\color{orange}Affect: “an instrumentality of intensities that may be attached to an array of phenomena and which may, depending on the intensity of these attachments, align other affective capacities of/for both object and subject alike. This conceptualization of affect is situated within broader transdisciplinary engagements with affect theory across the humanities, social sciences, and philosophy, where affect has emerged as a contemporary placeholder for ‘the social’ in the context of the ‘changing confunction of the political, the economic, and the cultural in which this cofunctioning is itself rendered affectively as change in the development of a reciprocal capacity to simultaneously affect and be affected.” (Careful with quoting here, sub quotations included. See original doc for reference).\cite{Leszczynski2019}}\\
%%-{\color{orange}“mobilizing geolocation as an entry point for both engaging with and substantiating a conceptualization of platforms as affective.”\cite{Leszczynski2019}}\\
%-{\color{orange}“geolocation organizes affects for platforms by investing users within platform ecosystems through individual and collective attunements towards using, participating within, contributing to, remaining within, and returning to these ecosystems.”\cite{Leszczynski2019}}\\
%-{\color{orange}“The integration of geolocation in the form of positional awareness as a native component of digital platform design invests individuals in highly structured yet simultaneously ontogenetic hyperlocal microgeographies that solipsistically spatially and temporally centre the user.”\cite{Leszczynski2019}}\\
%-{\color{orange}“These space-times invest users within platform ecosystems by literally positioning them at the centre of their own platform-mediated universe, affectively attuning users towards returning to the platform to regain access to a service brokered solipsistically to their hyperlocalities and real-times.” \cite{Leszczynski2019}}\\
%-{\color{orange} “Certainly, the instantaneousness and hyper-locality of platform mediation is significant to the competitiveness of platforms such as Uber in increasingly crowded platform economy marketplaces.”\cite{Leszczynski2019}}\\
%-{\color{orange}“beyond understanding these space-times in terms of the business strategies that they may underwrite affect theory also capture the ways in which once users are affectively invested in these ecosystems via geolocation-organized space-times of the ridehailing platform, these platform-mediated affects ‘turn back’ on themselves… organizing an affective frame of reference which conditions user’s expectations of future experiences of the platform: a potential ride available frome very possible location within minutes of being requested.”\cite{Leszczynski2019}}\\
%-{\color{orange}Amazon prime: “the ability of platform enterprises to operationalize these space-times for consumers through drone delivery is contingent on a complex of geolocation affordances, data, hardware components, and software technologies operationalized in concert,... also expands the range and possibility of modalities through which Amazon may organize affect that retain consumers within its ecosystem… through the leveraging of geolocation to organized the desired hyperlocal rea-times that align both online and offline consumption experiences with the expectations of consumers.”\cite{Leszczynski2019}}\\
%-{\color{orange}Geolocation as “an affective capacity as an instrumentality that organizes and aligns experiences of platform ecosystems. This especially pertains to consumptive experiences (of goods and services), which are affectively smoothed through the leveraging of geolocation as a coordinating force that produces seamless digital consumption experiences not only within but also across platform ecosystems.”\cite{Leszczynski2019}}\\
%-{\color{orange}“Streamlining the retail consumer experience across all stages of a purchase, from shopping online through to last mile fulfillment… here, geolocation is leveraged as a real-time coordinating force that links and streamlines multiple stages of an online-to-offline retail transaction so as to render the consumption process as experientially seamless as possible for the end consumer.”\cite{Leszczynski2019}}\\
%-{\color{orange}“Geolocation is also leveraged to organize affects of smoothness across platform an application ecosystems through practices of digital aggregation.”\cite{Leszczynski2019}}\\
%-{\color{orange}“By integrating multiple urban transportation asset providers, infrastructures, and operators within a single app ecosystem, Transit smooths urban mobilities by investing users within a singular, cohesive, integrated digital platform experience that mitigates the impracticalities of switching between multiple branded apps of mobility actos both across transportation modalities… and within a modality class.”\cite{Leszczynski2019}}\\
%-{\color{orange}“Inarguably, digital location has become central to quotidian experiences of digital platforms, be it through the ways in which content is accessed an interacted with (location-aware search, spatial interfaces); the ways in which we generate and contribute content (geotagging, personal location data mining); the ways in which we rely on location-aware devices such as sensors and digital assistant to produce ‘smart’ environments for us (smart cities, smart homes); and also the ways in which we interact with the world (augmented and virtual realities)... DIgital location is also an affective instrumentality that organizes how platforms and platform ecosystems are encountered and experienced by end users, and where thesmoothness of these experiences - as underwritten and facilitated by geolocation - is central to the ways in which users themselves become affectively invested in platform ecosystems, coming to see platforms as indispensable to how they get around cities. “\cite{Leszczynski2019}}\\
%-{\color{orange}“geolocation also functions as a techno-discursive instrumentality that underwrites the affective expression and enactment of social, familial, and romantic relations of connectedness performed through the sharing and monitoring of personal locations via digital platforms and their affordances of connectivity.”\cite{Leszczynski2019}}\\
%-{\color{orange}“digital location is itself also being leveraged as an instrumentality of securing trust in platform ecosystems.”\cite{Leszczynski2019}}\\
%-{\color{orange} “Mobilizing an understanding of location as implicated in arranging transparency and security form platforms informs a nuanced understanding of how it is that users become invested in platform ecosystems despite the ways in which platforms themselves undermine user confidence in digital systems through practices such as psychometric profiling and microtargeting..., covert user experimentation…, and regular data security breaches, including those involving the theft of personal location data”.\cite{Leszczynski2019}}\\
%-{\color{orange}“As an affective capacity to invest users in digital ecosystems, geolocation is leveraged to build trust in platforms by organizing technics and discourses of authorization (digital transactions), and of securitization (against misinformation).”\cite{Leszczynski2019}}\\
%-{\color{orange}Geolocation “invests developers, marketers, and entrepreneurs in the speculative potential of geolocation - in the form of embedded interactive map objects and interfaces - to organize annexation economies for end users. These attention economies are predicated on investing users themselves in digital platforms by ‘attracting eyeblalls’, expanding opportunities for user engagement by providing additional interactive elements (interactive map objects), and extending the duration of user visits (how long users remain within a digital ecosystem).”\cite{Leszczynski2019}}\\
%-{\color{orange}“geofilters are a brand growth strategy realized mobilizing real-world location… Here again digital location is being discursively mobilized to affectively attune digital developers and advertisers to the capacities of geolocation to both organize spatio-temporally contingent user attention economies and to ultimately convert that attention into both online and offline consumption.”\cite{Leszczynski2019}}\\
%-{\color{orange}“Geolocation expresses a capacity to align and attune corporealities and inorganic materialities in ways that work to invest users in platform ecosystems by organizing and orienting affects for platforms.”\cite{Leszczynski2019}}\\
%-{\color{orange}“ecosystem refers broadly to interconnected stakeholders and organizations exchanging knowledge and value, each with their own role in the greater whole. No one, central party controls the ecosystem, but rather, it is mutually nurtured and supported for collective health.”\cite{WEF2021}}\\
%\begin{itemize}
%	\item Uber
%	\item AirBnB
%\end{itemize}
%
\subsubsection{Location extraction from text}
%-{\color{orange} IDentify place name: Stanford Core NLP, OpenNLP; Identify placeal nd disambiguate: CLAVIN, Yahoo placemaker; Geography3 \cite{James2020}}\\
%-{\color{orange}“Extracting location names from informal and unstructured social media data requires the identification of referent boundaries and partitioning compound names. Variability, particularly systematic variability in location names challenges the identification task.”\cite{Al-Olimat2018}}\\
%-{\color{orange}Location name contraction problem: “pragmatic influences on writing style shorten names to reduce redundant content in social media.”\cite{Al-Olimat2018}}\\
%-{\color{orange}Nameheads: “complex phenomenon of alternate name forms”.  “1) Appellation formation, 2) explicitly metonomy, 3) category ellipsis, and 4) location ellipsis).” Appellation formation: removing the designator of a specific thing (O PONTE vs. PONTE DE 25 DE ABRIL). Explicit Metonomy removes the thing and leaves only the designator (NOVA vs. UNIVERSIDADE DE NOVA). “Both Appellation formation and metonomy pose disambiguation problems, and require context such as the author’s location to resolve.” “Category Ellipsis and Location Ellipsis pose delimitation problems that can be resolved with a statistical language model.” Category Ellipsis: “the author strips words related to the location category” (LISBOA vs. CIDADE DE LISBOA). “Location Ellipsis occurs when an author drops the specific location reference in the location name”. Entity delimitation: “to identify the boundaries of a location mention in the text”. Previous efforts invoke the application of heuristics (syntatic and semantic) with fallible results, challenge linking to gazetteers. Collocations: “a sequence of ordered words… neither strictly compositional nor always atomic.” “Given a region-specific gazetteer, which retains the same location-context as the text, we can construct a statistical model of the token sequences it contains.”
%\cite{Al-Olimat2018}}\\
%-{\color{orange}“Tokenize all location names in the gazetteer to construct the n-gram model and then save the resulting lists of unigrams, bigrams, and trigrams.” From the gazetteers, try to predict all the different ways one might reference location (assumed shortening of the official names) and create a new list.\cite{Al-Olimat2018}}\\
%-{\color{orange}“The field of geolocation extraction collectively involves many different tasks and analyses to be performed over text. The three main tasks among these are: (i) Location named entity extraction; (ii) Location named entity resolution; (iii) Event’s location extraction.”\cite{Imani2019}}\\
%-{\color{orange}Even though several geoparsers such as Cliff-Clavin, Mordecai, and Stanford-CoreNLP have been developed to automatically extract named locations from unstructured English text, location extraction from a text is still a challenging task due to the complexity, diversity, and ambiguity of location information in different languages. However, these tools cannot extract the focus location with good accuracy, and most of them cannot differentiate between different locations in the text - i.e. focus locality versus non-focus locality - and are not language agnostic.”\cite{Imani2019}}\\
%-{\color{orange}“While traditional named entity taggers are able to extract geo-political entity and certain non geo-political entities, they cannot recognize precise location mentions such as addresses, streets and intersections that are required to accurately map the news article.”\cite{Gupta2020}}\\
%-{\color{orange}Named Entity Recognition (NER) taggers: “contain tags to identify organizations, geo-political entities (GPE) and certain non-CPE locations such as mountain ranges and bodies of water from text”; “they are not able to extract precise locations mentions such as addresses, streets or intersections in their entirety.”\cite{Gupta2020}}\\
%-{\color{orange}“Tools trained on conventional NER models…. Have been successful in identifying common named entities. However, challenge comes when high level of granularity is of interest in extracting location entities such as specific addresses, streets, or intersections.”\cite{Gupta2020}}\\
%-{\color{orange}“Fine-tuning pre-trained language model fo domain-specific machine learning tasks has become increasingly convenient and effective”.\cite{Gupta2020}}\\
%-{\color{orange}Process: “named-entity tagger for the task of precise location extraction involves fine-tuning an existing neural network on a target dataset.\cite{Gupta2020}}\\
%-{\color{orange} “Geocoding is the process of taking input text, such as an address or the name of a place, and returning a latitude/longitude location on the Earth’s surface for that place.”\cite{Gupta2020}}\\
%-{\color{orange}Geotagging: 1. Entity feature vector extraction, 2. Gazetteer record assignment, 3. Geographic name disambiguation/toponym resolution, 4. Geographic focus determination\cite{Teitler2008}}\\
%-{\color{orange}“Unlike the spatial information used in a Geographic Information System (GIS), spatial information obtained from web documents is often incomplete and fuzzy in nature. A GIS user can formulate data retrieval queries specifying complex spatial restrictions, while a search engine targets a wide variety of suers who only provide simple queries. However, these users are also in need of retrieval mechanisms for queries with geo-spatial relationships. In order to support this, a first step concerns with assigning geographical scopes to web resources, so that the same resources can latter be retrieved according to geographical criteria.”\cite{Silva2006}}\\
%-{\color{orange}“Machine learning provides effective techniques for text classification, involving the automatic generation of classifiers from manually annotated training data. However, with very few exceptions, most work in automated classification has ignored the presence of hierarchically structured classes and/or features.”\cite{Silva2006}}\\
%-{\color{orange}“Much of the contextual information that could be used to disambiguate the geographical scopes in natural language texts is absent or external to the texts. The amount of training data per feature is also so low that there are no repeatable phenomena to base probabilistic methods on. For instance, the frequency of location names is in itself not sufficient for a good classification, as the same location name will usually not be repeated, even if the name is important.”\cite{Silva2006}}\\
%-{\color{orange}“Disambiguating geographical references in the text and assigning documents with a corresponding geographical scope are two crucial steps in building a geographical retrieval tool.”\cite{Silva2006}}\\
%-{\color{orange}“Previous studies have demonstrated that recognising geographical place names in text (usually called geo-parsing) is a crucial precondition for geo-referencing web documents. In language processing, the task of extracting and distinguishing different types of entities in text (i.e. names of people or organizations, dates and times, events, geographic features or even ‘non entities’) is referred to as Named Entity Recognition (NER).”\cite{Silva2006}}\\
%-{\color{orange} “Ambiguity is the main problem associated with geographical references in text… ambiguity in geographical references is bi-directiona, as the same name can be used for more than one location (referent ambiguity), and the same location can have more than one name (reference ambiguity). The former has another twist: the same name can be used for locations as well as for other class of entities, like persons or company names (referent class ambiguity).”\cite{Silva2006}}\\
%-{\color{orange}“To be useful, NER systems focusing on geographical concepts should handle the complex issues related to how people use geographical references. Place names lack precision in their meaning, and often vary with time, from person to person, and with the context in which they are used. Many times place names are simply temporary conventions, and people’s vernacular geography if also often vague, as they may also be interested in the vicinity of a place without knowing its exact name. Not only spelling variation are common on geographical names, but also the places those names reference change in shape and size.”\cite{Silva2006}}\\
%%-{\color{orange}“The GIPSY system for automatic georeferencing of text uses a disambiguation method that incrementally constructs a polytope via merging lat polygons, in such a way that a third dimension is introduced for the intersecting area (polygon stacking).”\cite{Silva2006}}\\
%-{\color{orange} “About 60\% of all data (textual and otherwise) are geospatially referenced”.\cite{Karimzadeh2019a}}\\
%-{\color{orange}Geographic information retrieval: GIR\cite{Karimzadeh2019a}}\\
%-{\color{orange}OVerall, soem recurrent issues persist in geoparsing research: methods are not evaluated or evaluated on non-public datasets or using proprietary systems (inaccessible, or behind paywalls): demonstration and evaluation rely on small tasks, or small gazetteers, or small geographic scope: and/or focused harvested corpora are used in evaluations that greatly simplify the toponym resolution task.”\cite{Karimzadeh2019a}}\\
%-{\color{orange}In unstructured, textual data: “the spatial data is specified using text (called toponyms) rather than geometry, which means that there is some ambiguity involved.”\cite{Lieberman2010}}\\
%-{\color{orange}“The ambiguity has an advantage in that from a geometric standpoint, the textual specification captures both the point and spatial extent interpretation of the data (analogous to a polymorphic type in parameter transmission which serves as the cornerstone of inheritance in object-oriented programming languages). On the other hand, the disadvantage is that we are not always sure which of many instances of geographic locations with the same name is meant”\cite{Lieberman2010}}\\
%-{\color{orange}Geotagging: “The process of identifying and disambiguating references to geographic locations (i.e., toponyms), known as geotagging, consists of two steps: toponym recognition, where all toponyms (e.g., “Paris”) are identified, and toponym resolution, where each toponym is assigned to the correct geographic coordinates among the many possible interpretations (e.g., “Paris” which can be one of over 140 places including France and also Texas). Geotagging is difficult because the first step involves understanding natural language, while the second step requires a good understanding of the document’s content to make an informed decision as to which of the many possible locations is being referenced.”\cite{Lieberman2010}}\\
%-{\color{orange}Toponym recognition: “the most common strategy is simply to find phrases in the document that exist in a gazetteer, or database of geographic locations, and many researchers have used this as their primary strategy.”\cite{Lieberman2010}}\\
%-{\color{red}Section 2 for shortcomings of current geotagging strategies\cite{Lieberman2010}}\\
%-{\color{orange}Fuzzy geotagging: “does not fully resolve toponyms in a single article, instead returning sets of possible interpretations for ambiguous toponyms.''\cite{Lieberman2010}}\\
%-{\color{orange}“Geocoding is the process of parsing places and addresses written in natural language into canonical geocodes, i.e., one or more coordinates referring to a point or area on earth.”\cite{Hamborg2019}}\\
%-{\color{orange} “Geographic description in texts reflect human conceptualization and experiences of space and places. Different from other forms of geographical data, text-based spatial description are subject to all sorts of ambiguities that prevent effective use.”\cite{Cai2016}}\\
%-{\color{orange}“Geospatial referencing textual documents refers to the task of discovering location phrases and creating unambiguous representation (or footprints) of the meaning of those textual references.”\cite{Cai2016}}\\
%-{\color{orange}“To create unbiased corpora that capture the complexity of natural language and place name ambiguity and annotate with ground-truth toponyms from high coverage and detailed gazetteers, documents used for testing and training should be manually ‘geo-annotated’ by human annotators, i.e., place names should be recognized (segmented) and manually resolved to toponyms in gazetteers. The process of manually tagging (segmentation) and annotating place names in text with entries (toponyms) from a geographic gazetteer, here called ‘Geo-Annotation’, is laborious, costly, time-consuming and error-prone. The scarcity of publicly available geo-annotated corpora can partially be attributed to the lack of available efficient software infrastructure capable of facilitating this laborious task.”\cite{Karimzadeh2019}}\\
%-{\color{orange}“Extracting the ‘correct’ location information from text data, i.e., determining the place of event, has long been a goal for automated text processing.”\cite{Lee2019}}\\
%
\subsubsection{Event extraction}
%-{\color{orange}“Event extraction is the process of recognizing defined event types in text (e.g. “attack” or “protest”) and extracting and classifying the actors involved in the events.”\cite{Halterman2019}}\\
%-{\color{orange}“To be useful in subnational research, these events require information on the location where they occurred. A second related information extraction task is ‘geoparsing’, the process of recognizing place names in text (‘toponym recognition’) and resolving them to their coordinates or gazetteer entry (‘toponym resolution’).\cite{Halterman2019}}\\
%-{\color{orange}Tokenize a sentence. Potential for multiple events per sentence, potential for zero, one, or multiple tokens per event. Tokens labeled as 1 (if location of event) or 0 (otherwise). Assumes “that events are ‘anchored’ by a verb, is a common assumption in semantic role labelling, a closely related task to event-location linking”. \cite{Halterman2019}}\\
%-{\color{orange} “Many existing open source geolocated event datasets, including GDELT and Phoenix, make no effort to explicitly link events and locations, simply returning a top location from a sentence, without using information on the extracted event to inform the geolocation step, which has also been used in NLP”.\cite{Halterman2019}}\\
%%See ``Previous Work'' section of Halterman2019
%-{\color{orange}“Primary focus location” of Imani (2017): “makes the simplifying assumption that documents have one single, fixed ‘focus location’ that is invariant to different potential events in the document.”\cite{Halterman2019}}\\
%%-{\color{orange}Lee (2018): “model is only able to located events to governorate/province (ADM1) level, and finds locations based on a dictionary search of known place names”\cite{Halterman2019}}\\
%-{\color{orange}“While humans are able to pick up on nuance and deal with grammatical complexity that machines still cannot handle, humans are also unsuited to the tedium of labeling thousands of sentences and may be susceptible to drift in their definitions or understanding of the task. Not only is the automated method vastly cheaper and faster than the human process, it does so with accuracy at least as good.”\cite{Halterman2019}}\\
%- {\color{orange} “Text also holds a great deal of factual information and new techniques are needed to allow researchers to extract political information from text.”\cite{Halterman2019}}\\
%-{\color{orange}“Many quantitative studies of conflict rely on event data. Recently, these studies have also retreated from the country-year framework and have focused on disaggregating the event flows both in terms of space and time. DIsaggregating temporality - even to the daily level - is a straightforward task. But figuring out precisely where an event actually occurred is a difficult and uncertain task that has been perplexing for most contemporary event data efforts”.\cite{Lee2019}}\\
%-{\color{orange}“Most data in the conflict realm comes from non-official sources. FOr many that means some form of data collected form historical and journalistic sources. This need is often filled by event data, which are typically collected on a daily basis, and can be aggregated temporally to the level required by the analysis. Even data can also be aggregated to the geographical region that is appropriate. Given the increasing demands fro event data, the scientific community has recently devoted significant efforts to automate the data collection process. Having humans read and code a large set of archive documents sometimes limits reproducibility, and hence hinders scientific research. It is also expensive and limits the currency of the data. Further, ensuring inter-coder reliability is challenging, especially over global events that span decades.”\cite{Lee2019}}\\
%-{\color{orange}“These automated event data allow researchers to observe and extract information on politically relevant events around the world in near real-time.”\cite{Lee2019}}\\
%-{\color{orange}“Outstanding issues [in automated data collection] include machine translation of texts in foreign languages (wherein great progress is being made in both Chinese and Arabic), duplicate reports from multiple sources, and the relatively low accuracy in determining the event location.”\cite{Lee2019}}\\
%-{\color{orange}“Named entity recognition in the context of geolocation involves determining which words in the given sentences are location names. In principle, the task of capturing location names from texts can be done easily by using a dictionary. In practice, however, developing a dictionary that is sufficiently comprehensive for such a task may be challenging. To begin, the geographic boundary of the texts being analyzed may be unclear, given that the domain of many even data is the entire world. Further, because conflict events often spread to new and rural places, texts may include location names not defined in the gazetteer. Still further, a location name may be written in multiple forms, requiring the dictionary to comprise every variant for each location.”\cite{Lee2019}}\\
%-{\color{orange}Ground truth: hand coded set of actual locations for training and verification\cite{Lee2019}}\\
%-{\color{orange}“The task of determining event locations involves three steps, each non trivial.” 1. named entity recognition (NER): “all location names are identified and extracted from an appropriately preprocessed text. This step is a prerequisite for the other steps because to determine the location of an event in a news article, capturing the exhaustive list of location names is required.” 2. Ambiguation/resolution: “which involves identifying the actual location of the recognized name string. Once this is accomplished, it is possible to extract the ontologically defined meaning from the text in terms of who does what to whom, and when and where”. 3. Determine if disambiguated location names are the event location.\cite{Lee2019}}\\
%-{\color{orange}“geolocating events (identifying the location of the event described in a document) is an objective for many scholars, particularly those who intend to collect and build original databases from text corpora, be they news articles, congressional records, campaign speeches, party constitutions, or twitter feeds. While automating this task will aid many, the research avenue in this topic is still under development.”\cite{Lee2019}}\\
%-{\color{orange}“Given that a substantial number of location words are incorrect event locations, the automated event data community needs a better coding scheme that can reduce the error rates.”\cite{Lee2019}}\\
%-{\color{orange}Event-relevant: “those locations that are part of the main description of the event of interest, i.e., all locations that are key to the narrative of the event of interest.”\cite{Lee2019}}\\
%-{\color{orange}Event-occurring: “all locations where events occurred regardless of whether the event is the event of interest.” \cite{Lee2019}}\\
%-{\color{orange}Eventirrelevant and event occurring location: “such could occur when the raw texts contain news summaries of events that are not of interest.”\cite{Lee2019}}\\
%-{\color{orange}“An N-gram is a sequence of N words. Collections of N-grams are known to provide valuable information about each word in a phrase, taking into account the complexity and long distance dependencies of languages… Given that the collocation patterns in which the event-occurring location words appear differ from those of the non-event-occurring collocations, teh N-gram patterns are able to provide the contextual information of even-occurrence to our classifiers.”\cite{Lee2019}}\\
%-{\color{orange}“Geolocation inference at the event level estimates the location of events mentioned in text. This level of inference predominantly relies on geoparsing  the process of identifying geolocations in text and disambiguating between multiple toponym references… event geolocation inference might not reflect the actual location of individual tweets.”\cite{Snyder2019}}\\
%-{\color{orange}User geolocation: “user locations can be predicted by utilizing toponym references within their tweets a well as user metadata such as friend networks and time zones.”\cite{Snyder2019}}\\
%-{\color{orange}Tweet level (article level): “estimates the location of individual tweets. THis differs from user-level prediction in that a tweet might be posted in a separate location from where they live, such as during a vacation or work hours.”\cite{Snyder2019}}\\
%
%Examples: 
%\begin{itemize}
%	\item Cliff-Clavin
%	\item Mordecai
%	\item Stanford-CoreNLP
%	\item BERT: {\color{orange}BERT-based language model. Google’s Bidirectional ENcoder Representations from Transformers, “powerful, pre-trained deep learning based language models”\cite{Gupta2020}}. -{\color{orange} BERT: “a general purpose language representation model pre-trained on millions of articles on English Wikipedia an BookCorpus.”\cite{Gupta2020}}
%	\item Wheb-a-where: {\color{orange}Web-a-Where: “This small size imposes a serious limitation on Web-a-Where’s practical geotagging capabilities, as it is unable to recognize the small, highly local places that are commonplace in articles from local newspapers.”\cite{Lieberman2010}}
%\end{itemize}
%
%Tumba
%\begin{itemize}
%	\item {\color{orange} Tumba: a web search engine for Portugal (Gomes \& Silva 2003)\cite{Silva2006}}
%	\item {\color{orange} “We propose to further improve the quality of search systems, by integrating the geographical knowledge that can be inferred from web resources.”\cite{Silva2006}}
%	\item {\color{orange}Scope: “the region, if it exists, whose readers find the page more relevant than average. A geographic scope is specified as a relationship between an entity on the web domain (a HTML page or a website) and an entity in the geographic domain (such as a location or administrative region). The geographic scope of a web entity has the same footprint as the associated geographic entity.” Geographic scope definition for reader engagement (not generically organizing location for research or general applications later\cite{Silva2006}}
%	\item {\color{orange}“Geographic information is pervasive on the web. An analysis of 3,775,611 pages 8,147,120 references to the 308 Portuguese municipalities (administrative division of the territory corresponding to populated sub-regions of the Eurstat NUT 3 area), an average of 2.2 references per document.”\cite{Silva2006}}
%	\item {\color{purple}Geo Tumba provides a retroactive organization and characterization of existing webpages, versus specifically defined ruing writing.\cite{Silva2006}}	
%	\item {\color{orange}“The Portuguese ontology included more place names, but results in terms of recall are inferior to experiments using hte smaller global ontology… results indicate that recall does not improve considerably with the amount of available place names.”\cite{Silva2006}}
%	\item {\color{orange}“Our framework differs on the emphasis put on geographic name entity recognition, the use of a graph ranking method for assigning a single scope to each document, the extensive use of names instead of coordinate information, and the availability of ontologies associating entities to geographic scopes.”\cite{Silva2006}}
%\end{itemize}
%
%LocateXT\\
%-{\color{orange}Search unstructured data for spatial locations\cite{ArcGisLXT}}\\
%-{\color{orange}Geneartes point features representing locations \cite{ArcGisLXT}}\\
%-{\color{orange}Attributes include file, context related text, dates, keywords \cite{ArcGisLXT}}\\
%-{\color{orange}Not addresses\cite{ArcGisLXT}}\\
%-{\color{orange}Custom locations: “When documents are scanned, they are examined for place names specified in a custom locations file. The custom locations file associates a place name with a spatial coordinate. A point is created in the output feature class to represent each location found.”\cite{ArcGisLXT}}\\
%-{\color{orange}fuzzy matching: applies to custom locations. 70\% of alphanumerics match\cite{ArcGisLXT}}\\
%-{\color{orange}Require word breaks setting “when word breaks are required, text is considered a word when it is bounded by whitespace or punctuation characters as in European languages.”\cite{ArcGisLXT}}\\
%-{\color{orange}Require word breaks setting “when word breaks are required, text is considered a word when it is bounded by whitespace or punctuation characters as in European languages.”\cite{ArcGisLXT}}\\
%-{\color{orange}Custom location: “is defined by associating text with a spatial coordinate. Current and historical places, and natural features and structures can all be defined as custom locations in a custom location file (.lxtgaz).”\cite{ArcGisLXT}}{\color{purple} Input your own gazetteer (custom or otherwise). Again, only points (not polygons).}\\
%
%-{Integrated Crisis Early Warning System (ICEWS)\cite{Lee2019}}\\
%-{Open Source Event Data Alliance\cite{Lee2019}}\\
%
%GeoTxt:
%\begin{itemize}
%	\item {\color{orange}a scalable geoparsing systems\cite{Karimzadeh2019a}}
%	\item {\color{orange}SPIRIT: spatially aware information retrieval on the internet\cite{Karimzadeh2019a}}
%	\item {\color{orange} Geographic footprint of documents\cite{Karimzadeh2019a}}
%	\item {\color{orange}Heuristics for toponym disambiguation: spatial proximity (“spatial minimality”), co-occurence (relation based on spatial hierarchy levels)\cite{Karimzadeh2019a}}
%%	\item {\color{orange}Local Global (LGL) project: “contains documents from local newspapers, each having a geographic footprint of an area proximate to the publication site.”\cite{Karimzadeh2019a}}
%	\item {\color{orange}Appropriate for hyperlocal applications: “advantageous on a corpus with a smaller known geographic footprint”\cite{Karimzadeh2019a}}
%%	\item  {\color{orange}“metonyms (such as “Toronto” just announced the dates of winter Olympics) and demonyms (such as “Americans” are thrilled with the news) are marked as place names, which we and several other researchers contend are wrong annotation decisions. For instance, the LGL corpus expects New York in “New York Times” to be resolved to the state of New York, whereas we consider “New York Times” to be the name of an organization (which can potentially resolved to the New York Time headquarters toponym if the context makes it clear that the headquarters building is being referred to, rather than the organization overall).\cite{Karimzadeh2019a}}
%\end{itemize}
%
%{DeepGeo \cite{Snyder2019}}\\
%
\subsubsection{Gazetteers}
%-{\color{orange} “The locale name for an object within a language area is termed an endonym. Outside this language area, te ame object may have other names according to the respective language. These variations are called exonyms.” \cite{Witschas2004}}\\
%-{\color{orange} “The United Nations Group of Experts on Geographical Names (UNGEGN) is run by specialists from the fields of linguistics, cartography and history. The UNGEGN requests national gazetteers (alphabetical lists of names, with coordinates and other data) to promote the use of nationally standardised names on maps and in written documents.”\cite{Witschas2004}}\\
%-{\color{orange}''An exonym is a place name that isn't used by the people who live in that place but that is used by ohers.''\cite{Nordquist2018}}\\
%-{\color{orange}''a locally used toponym - that is, a name used by a group of people to refer to themselves or their region (as opposed to a name given to them by others) - is called an endoynm.\cite{Nordquist2018}}\\
%-{\color{orange} Gazetteers assist in geographical entity identification performance\cite{Silva2006}}\\
%-{\color{orange} “The larger the gazetteer (in terms of coverage and detail), the harder the task of toponym resolution, since the gazetteer will contain a higher number of ambiguous names. However, a large gazetteer such as GeoNames anables resolving more places to toponyms.” \cite{Karimzadeh2019a}}\\
%-{\color{orange}Progressive geo-coding environment: “this configuration of geo-referencing system ensures that a local gazetteer is incrementally enriched through local community use and the demand for the effort of human go-coding is kept to the minimum.”\cite{Cai2016}}\\
%-{\color{orange}“Gazetteers are well-structured and maintained, but they have limited coverage on those less known places at local levels. GIS databases have geographical features in greater geometric precision and details, but they tend to have limited textual metadata for matching to place names. Open Source geographical information now comprises and ever growing part of geographical knowledge, and they frequently include vernacular descriptions of locations, as well as references to imprecise areas. The problem with using CGI is that their local coverage is uneven from locality to locality, and they tend to have various degree of quality and trust. Ideal VGI is supposed to be produced entirely by individual citizens that reside in their locality, and therefore potentially represent the best knowledge about their local.”\cite{Cai2016}}\\
%-{\color{orange}“Majority of gazetteer entries use points (coordinates) to represent areas (such as a city), which is considered inadequate for local features.\cite{Cai2016}}\\
%
%Gazetteer examples:\\
%-OpenStreetMap: {\color{orange} open data, licenced nder the Open Data Commons Open License (ODbL) by the OpenStreetMap Foundation (OSMF): \cite{OSM2020}}. {\color{orange}Relies on OSM “performed the best”... “DBpedia is not focused on geographical information; therefore, it does not contain the metadata useful for the system’s future use (e.g., extents and full addresses). Also, OSM has more fine-grained locations and more accurate geo-coordinates than Geonames”.\cite{Al-Olimat2018}}. Use \cite{Al-Olimat2018}\\
%- Geonet Names Server (GNS): {\color{orange} US NGA maintained: GeoNames. ``The GEOnet Names Server (GNS) is the official repository of standard spellings of all foreign geographic names, sanctioned by the United States Board on Geographic Names (US BGN). The database also contains variant spellings (cross-references), which are useful for finding purposes, as well as non-Roman script spellings of many of these names. All the geographic features in the database contain information about location, administrative division, and quality. The database can be used for a variety of purposes, including establishing official spellings of foreign place names, cartography, GIS, GEOINT, and finding places.'' \cite{NGA2020}}. Use \cite{Al-Olimat2018}; {\color{orange}“The GeoNames gazetteer was chosen for GeoText because of its extensive coverage, quality, inclusion of metadata (such as alternate names and geographic hierarchical information),and frequent updates”.\cite{Karimzadeh2019a}}; {\color{orange}Official sources include U.S. National Geospatial Intelligence Agency (NGA) and the U.S. Board on Geographic Names, “assuring extensive coverage and quality throughout the world.”\cite{Karimzadeh2019a}} {\color{orange}“GeoAnnotator uses GeoNames as its default gazetteer because of its frequent updates, ease of ability by years to correct information, extensive coverage and quality and inclusion of metadata items necessary for geoparsing/geo-annotation, such as alternative names and spatial hierarchies. All GeoNames toponyms have a unique identifier, enabling creation of corpora useful for linked data applications.” GeoNames: “the richest gazetteer available at the time of this writing.”\cite{Karimzadeh2019}}\\
%-{\color{orange}DBPedia\cite{Al-Olimat2018}}\\ Requires premium subscription for polygon access.
%-Getty THesaurus of Geographic Names (TGN): $http://www.getty.edu/research/conducting_research/vocabularies/tgn/$ “TGN is a structured vocabulary including names and associated information about both current and historical places around the globe.”\cite{Silva2006}\\
%%-{\color{orange}Portugal instance: http://xldb.fc.ul.pt/geonetpt as an OWL (Web Ontology Language) instance\cite{Silva2006}}\\
%
\subsubsection{Local lexicons}
%-{\color{orange}Automatic inference of local lexicons: “1) Stability: a Local lexicon is constant across articles from its new source. 2) Proximity: Toponyms in a local lexicon are geographically proximate. 3) Modesty: A local lexicon contains a considerable but now excessive number of toponyms.”\cite{Lieberman2010}}\\
%-{\color{orange} “a spatial lexicon can be classified as a local lexicon if and only if the toponyms within it are geographically proximate.”\cite{Lieberman2010}}\\
%-{\color{orange}“Associating a single local lexicon with each data source allows for a variety of applications. However, it may be possible to fine-tune the use of spatial lexicons in situations involving different types of content. For example, a blog may track several different topics simultaneously, and use different spatial lexicons for each oipc. Furthermore, individual authors may write for specific audiences as well, as in the case of journalists stationed in certain geographic areas and concentrating on stories in that area. It thus might be beneficial to determine separate spatial lexicons assumed by different authors, and further improve geotagging performance. Ore generally, we might associate a particular spatial lexicon with any type of entity found in each document, be they authors, persons, organizations, or particular keywords. For example, upon finding a mention of ‘Robert Mugabe’, we might assume a spatial lexicon including SZimbabwe and nearby locations, even without specific mentions in the text.”\cite{Lieberman2010}}{\color{purple}Complicated yet potentially effective alternative to manual association.\cite{Lieberman2010}}\\
%-{\color{orange}“It would also be interesting to detect and observe evolving spatial lexicons over time for data sources with evolving geographic interests, thus further improving geotagging on these sources. For example, the first few articles of an ongoing, prominent news story will often fully specify the toponyms relevant to the story. Later articles in the series, however, will often underspecify the same toponyms, since they have already been introduced into the audience’s spatial lexicon and been fully resolved in early articles.”\cite{Lieberman2010}}\\
%-{\color{orange}“As newspapers and other data sources continue to move into the virtual space of the INternet, knowing and using spatial lexicons will be ever more important. Previously localized newspapers will cater to a broader, global audience, and thus will adjust their notion of their audiences’ spatial lexicons, perhaps limiting or ding away with an assumed local lexicon altogether. On the other hand, as more and more people publish highly individual and geographically local content, inferring individual local lexicons will be a necessity for correct geotagging. Geotagging with knowledge of local lexicons will thus continue to play a large role in enabling interesting geospatial applications.\cite{Lieberman2010}}\\

