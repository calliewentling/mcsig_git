\subsection {Justification}
\subsubsection{Concept}
Apregoar is a foundational, proof of concept web application that seeks to demonstrate the possibilities of intentionally associated temporal and spatial attributes to news stories by media publishers for an improved user experience for traditional readership, as well as improved searching capabilities for researchers and informational dashboards for monitors. Through various views of the underlying geospatial database, a variety of users may interact with news media in new ways to glean additional insights about relevant places, histories, or previously obscurred geospatial patterns. Each of these functionalities, though intimately connected by a communal database and shared system architecture, provide such different user experiences that they can be considered different informational products. This proof of concept, therefore, focuses on only a subset of these to demonstrate the idea. At the conclusin of this work, the protoype will be available for initial testing by a variety of users, draw constructive criticism, and ultimately serve as a base from which future, more fully featured platforms may evolve. 

 It is expected that the association of specific place (potentially non-conforming to existing administrative boundaries or defined points of interest) to traditional news articles will provide an added dimension of understanding to communities at a local level.  This type of data preparation, though it is initially cumbersome to establish and requiring adjustment of publishers' processes to maintain, may provide a powerful foundation from which future economic (improved publisher products elevating their offering and attracting/maintaining a customer base), societal (illumination of local trends requiring intervention, improved community engagement of readers with their surroundings, or improved city resources), and academic (improved research functionality) benefits may stem.  If this type of functionality and improved user experience are well-implemented by a handful of productive news services, it could inspire a shift of the industry standard towards integration of spatial attributes and spatially related products. Of the key drivers of geoportal advances (including  scientific geospatial projects and applications and international organization) commercial and governmental drivers are most applicable to this project, as they have the greatest voice and interest in conveying more clearly the spatiality of happenings within a hyperlocal community \cite{Jiang2020}.%-{\color{purple}of the drivers of geoportal advances (scientific geospatial projects and applications, intenrational organization, governmental agenceis and commercial purposes), governmental agencies and commercial purpose are most applicable to this project\cite{Jiang2020}}\\


\subsubsection{Views}
\paragraph{Georeferenced Data Entry}
Publishers interested in harnessing the foreseen spatial benefits of geospatial news for their own research and readership experience may incorporate the georeferencing step into their publishing process. Though ideally developed in the future as a plugin to existing publishing processes (such as a wordpress plugin for organizations leveraging the NewsRoom tool), initially the corpora will be added separately through the portal itself. At this stage, publishers will enter standard article information such as title, summary, web-link to their hosted article, define a story type (report, article, editorial, etc.), section, and tags. Additionally, they will be prompted to associate one or more instances: temporal/spatial definitions of the events described within the article. The temporal element will be defined to the day, requiring a start and end day (which may be the same), as well as the opportunity to textually describe the interval as well.  To georeference the incident, they will have the opportunity to draw one or multiple polygons describing the area where the events occured. If appropriate, the publisher may define multiple incidents, if multiple events occured or they are describing impacts that changed over time.

In future variation, publishers should be able to "whitelabel" their entries, creating a platform that they can integrate into their own websites that include only their own publications, as well as additional features such as recommending spatially or temporally similar articles to readers. These features, much like the now-common features of recommended stories by theme already integrated into may digital news sites, may support additional readership engagement with the site.



\paragraph{Spatial News Database}
The publishers' entries will form a foundational database that will be accessed by each informational product and applied to the greater metropolitan area. This forms the beginning of a geo-annotated corpus of local news stories that can be used both by the Apregoar tool, as well as extracted by users for inclusion in other projects or studies. %-{\color{purple} Both a tool and the beginning of a database that can be used and applied in the Lisbon area.}{\color{orange} Focus of both “producing a geo-annotation software platform, but included a proof-of-concept effort to build a manually geo-annotated corpus along the way.”\cite{Karimzadeh2019}}\\
The relational database should store each story with incidents as a one to many relationship. 

Additional operational information, such as user profiles, should also be associated to ensure appropriate recall of stories by publisher, or any personal preferences or history users may elect to set.

\paragraph{Contextualization Maps}
It is understood that the integration of visualization into existing products is an important part of relevance and commercial value propositions\cite{Meeks2019}. Therefore, a key view that can immediately improve the user experience of local news readership is the integration of local maps associated with even a single story that can provide additional spatial context to a publications readership. %-{\color{purple}Integration of visualiztion into existing products is an important part of relevance and commercial value propositions\cite{Meeks2019}}\\
In this view, a page dedicated to a single news event is loaded, with critical data such as title, summary, publication date, author, etc. along with a map situating the associated georeferrenced incidents. Users may use the basemap points of interest or scrolling functionality to oriente themselves to the location of the stories events.

In future iterations, nearby events may also appear in a different color to indicate recommended and spatially relevant stories.

\paragraph{Searching Maps}
News stories are used by traditional readership to become better informed on what is happening in the world or in one's own backyard. If one is interested in entire swaths of areas, such as city wide, national, or international news, they may not care to discern between stories that happen in particular locales. Other readers, however may be more selective and particularly interested in things that are occuring near them. They may be interested in news that happens near where they live, study, or work, and their commutes between them. They may also be interested in news that occurs near family members or friends, but not much for whatever happens between. These users may be interested in identifying news only in relevant locations that don't adhere to administrative boundaries.

It may also happen that one has experienced an unplanned incident (such as saw a fire or an accident), or notice of a planned event and wants to understand more about it. To learn more, however, can be challenging if not already acquainted with the subject matter or particular name of the occurance. As of yet, news platforms don't integrate spatial searching beyond the incorpoation of keywords, or perhaps the choice of entire municipal areas.  Likewise, most temporal search is limited to the publication date.

Stories are also leveraged for research purposes to understand things that have already happened in the more distant past for academic or operational understanding. In these cases, searching for such incidents may be particularly challenging as names of places may have evolved over time, or have colloqual titles that the researcher is not privvy to.

In any of these scenarios, readers or researchers may be interested in the opportunity to define time intervals of the events of the news, as well as define their own area to return a map of associated stories (also filterable by their other thematic attributes) to better direct their browsing or searching experience.

This kind of temporal and spatial searching has already been implemented into many of the applications used regularly all over the world, as described in {\color{red}section of existing applications}. 

\paragraph{Monitoring Dashboards}
Though not yet implemented, monitoring dashboards are one of the most interesting potential applications of such georeferenced data. These provide the opportunity to layer geointelligence into the platform, supporting decision makers to take action based on the produced findings.

Users of the monitoring dashboard may leverage a variety of statistical techniques to better understand the ground tuth, especially density estimation and analysis of spatial distribution. Perhaps, this can also eventually leverage other interesting infromation, such as demographic, crime, or weather data as potential additional features. However, other interesing analytical methodologies should leverage the data extraction functionality of the tool (future integration of download or API connections) to layer this georeferenced communication data into systems more appropriate to handle this kind of data manipulation and yeild additional geointelligence insights.

\subsubsection{Nuanced place}
Place is hard to pin down. Geographical cue words such as ('city of Lisbon', 'just outside of Lisbon', 'Lisbon-based', and 'Rio Tejo' , adapted from \cite{Lieberman2010}) don't have precise spatial definitions, and can queue different understandings in various contects. %-{\color{orange}Geographical cue words. “‘City of X’, ‘just outside of X’, ‘X-based’, and ‘X Ocean’” \cite{Lieberman2010}}
"Just outside of Lisbon" is a particular area near Lisbon, but likely not a buffer of the city, more likely its an area to the side. Rio Tejo shouldn’t necessarily associate a point to the center of the entire of a river flowing thorugh the entire country, but is rather relative based on the context of the story/place (an area of river just off the coast of a particular place, for example). In these cases, there isn’t an appropriate automatic method to distil this from textual description.%\cite{Lieberman2010}\\
News articles also depend on nearby points of interest to situate their story, instead of using formal names. Though this may help to cue spatial contextualization in readership, it depends heavily on readers already having an understanding of the spatial layout of an area \cite{Lee2019}.% -{\color{orange}“Further complicating matters, new articles often use nearby landmarks to indicate the location, in lieu of using the official names.”\cite{Lee2019}} {\color{purple} Though for my purposes this is better, the sub-specific is helpful in sub-city level definitions. Of course, this does lead to challenges in places being defined by near-by POIs when they occur within, encompassing, or nearby (not exactly to the extent of the POI itself).\cite{Lee2019}}\\ 

Where something is happening may be best associated with existing gazetteers of information. These are expected to be point data (addresses, POIs, etc.) or polygons (administrative boundaries). There is a value to leaving the data as is (premitting the publisher to define a point or a polygon), that can be transformed on the fly as necessary \cite{Brown2012}. However, as nothing happens in a location of zero dimensions, the tool gently pushes the user to define a polygon, that provides a consistent experience and can still be transformed as necessary. %These can be left as-is (incorporating both types of data). The tool should allow the publisher the flexibility to define a point or polygon (more likely) location based on the “exact” occurrence of instance. Any data analysis from there can be transformed on the fly as necessary.\cite{Brown2012}\\
In cases in which points tend to be more appropriate, users are invited to draw sufficiently small polygons, representing a single building or even a sub area of this. %-{\color{purple}Having the ability for contributors to decide which is more convenient with quick assignment tools may be critical for user buy-in}{\color{orange} “we suggest tha polygon methods appear better suited for structured interviews, group-administered surveys, or workshops that provide face-to-face support for completion, whereas points may be a better choice for self-administered surveys.”\cite{Brown2012}}\\
By requiring custom areas (which can use existing boundaries as templates from which to draw) it also pushes publishers to think beyond existing areas and to carefully consider whether indeed their spatial description applies to entire administrative boundary, or perhaps is less completely and binarily affected across that space. It shoves off some of the rigid definitions already assigned to areas that fall within physically or politically defined areas, and permits new identities to formulate across or within these predefined areas \cite{McQueenBaker2019}. %-{\color{orange}“ A map, relied upon for accuracy in terms of direction and location, is a definitive symbol of capital “T” truth. As such, geographic boundaries, both physically and politically constructed, powerfully separate and shape the identity of those bounded by being located or represented on a map.” Validating incorporation of custom place definition\cite{McQueenBaker2019}}\\

Just as a reader’s lexicon develops over time, so should the associated gazetteer.  Manual specification allows the user the opportunity to name/address and define these flowing areas as their and general understanding of an event’s location changes over time. A military base may change and move over time, the same named thing within a city could come to mean different locations at different times. The flexibility to adjust on the fly is critical to the success of this type of initiative.\cite{Lieberman2010}.  These custom designed gazetteers, beyond contextualizing the news and providing a spatial database of associated place, can also be used in and of themselves to explore the patterns in understanding of how a place is named, external to any event that has occured there. For example, an area may be frequently referred to colloquially with one name, though technically it may be associated to another place. This element provides an indirect opportunity to study placemaking within the study area, external to any official news communication.

\subsubsection{Application in participative communities}
The tool can be considered a PPGIS system in that those assigning spatial definitions will not be trained GIS users. Rather, they will mostly be composed of jouranlists and publishers and officials or public institutions, though an even more broad definition of "public" is posible as non-institutionally affiliated users are also welcome to contribute their own georeferenced stories.  It can also be considered a PPGIS as it will required the assigned of place on both objective place definitions (addresses or existing boundaries) as well as custom definitions (areas that don't conform to administrative boundaries, understood point of interest, or incorporate multiple areas) \cite{Brown2012}. The visualization of this data is then available to any user for further exploration at will. %{\color{purple}Spatial data agreement only applies to drawing inferences (statistical analysis to derive information) from the system, considering journalists/publication and officials/public institutions as ‘public participators’\cite{Brown2012}}\\

Its greatest value, however, is as an input to a PPGIS, helping to form a general spatial understanding of the public by providing additional insights to citizens about the areas that affect them -- where they live, work, play, or have an interest. %\cite{Evans-Cowley2010}\\
As a dynamic evolution of placemaking within a local community, the tool aggregates and georeferences the city’s own public information onto a navigable map, but can invite and accommodate commercial information (online publications) as well as private citizens micro-blogs (less structured and potentially long form versions of “Na Minha Rua”) to gain further citizen feeling or commentary about their physical surroundings. %\cite{Evans-Cowley2010}\\

%\subsubsection{Geointelligence}
%Much like other information products, the tool would provide a platform for dissemination of knowledge and providing geo-referenced articles from which further information development is required for various purposes, and additional tools are required for further action. %\cite{Evans-Cowley2010}\\\\
%The data is also available openly for free utilization by the public for any desired purpoes, participative or otherwise. 
%
%Geographic information Science (GISc) can be enhanced by the location of incident reports (news articles).  Beyond colloquial data (social media), geolocation of official reports can contribute to both how an event is felt within and outside of a community. The differences between styles of reporting can be evaluated from place to place, as well as the relative importance given to a place by various communities\cite{Datta2018}. %-{\color{purple}Use reporting of news (both the reporting of an event as well as the time and place of the contents) with continuous sensor 
%When paired with other data sources suches as IOT or VGI, the impact of any given event may be better understood for its influence on others, such as reporting an election's impact on societal activity or the environmental accelerometer data pre-coverage of an earthquake \cite{Bhattacharya2018}.%data (such as physical sensors or VGI) to see impacts and trends. Ex: reporting of an election’s impact on societal activity or environmental accelerometer data pre-coverage of an earthquake}\cite{Bhattacharya2018}\\
%It may also be possible to better detangle cooccuring events and the possible impacts that they have on each other, such that the impact of one event isn't misattributed to another simultaneous one.
%
%The tool should support situational awarness via dynamic dashboards\cite{Varanda2020}.%-{\color{purple} the tool should support situational awareness via dynamic dashboards \cite{Varanda2020}}\\
%Everyday users are becoming increasingly exposed to and interested in geopolitics\cite{Granger2020a} as well as datal visualization tools. Both of these have been apparant during the COVID-19 pandemic news coverage that inherently has a spatial component, which has been the study of many great minds as future scenarios are predicted and immediate preventative measures have been enacted at every level of community, from the parish to the international regional level.  %-{\color{orange}Geopolitics and immigration: “subjects that 2020 has pulled into sharp focus” Importance of and interest in geopolitics for everyday users \cite{Granger2020a}}\\
%Such geoportals can, then, help any type of user better identify the data they require and notice trends among seemingly dissparate data records \cite{Jiang2020}.%-{\color{orange}“The geoportals can facilitate researchers, government officers, and ordinary users in helping them to find the data they needed, with basic searching services equipped in the geoportal.\cite{Jian2020}}{\color{purple}General theoretical support for geoportals supporting multiple types of users, so as in the publication geoportal tool.\cite{Jiang2020}}\\%


\subsubsection{Distinction from previous work}
Unlike many of the example provided in the literature, this effort seeks to georeferrence articles instead of social media posts, most noteably "tweets".  Tweets or other specific data types include structured organizations and data APIs from which automated programs attemp to derive sentiment and/or relation to particular events \cite{Snyder2019}.\\

The movement to incorporate citizens as sensors is important and powerful, but we have jumped over public and commercial sources of information that can not only contribute to but contextualize citizen feeling towards a place. Public and private data sources of events are being underutilized -- the content exists but needs to be georeferenced in order to be better accommodated by citizens, public management, or private enterprises. Citizen knowledge of a place is drawn from both anecdotal experience as well as learned (read) information from third person sources (news papers, reports, etc.). The association of place to news sources can be studied for its influence on public oppinion and determien how the spread of news affects public oppinion.\\

This project also fills a different niche than the projects attempting to parse a variety of information (spatial, temporal, and thematica such as sentiment, volatility, etc.) from international jornals and articles, While immensely valuable for a host of applications, the inherent nature of automation and post-processing (of articles or tweets) makes this method prone to inaccurate results. By incorporating human-in-the-loop association, this project seeks to develop a novel standard for allowing a publisher to explicitly assign temporal and spatial attributes to data records which can then be used as highly accurate input for future extrapolation of causality or trends within a community. The resulting data set should, therefore, avoid misassociation of time and place (such associating activity from Lisbon, Portugal to Lisbon, Ohio in the United States of America, or accurately differentiating between the Distrito de Lisboa, the Município de Lisboa, and the Área Metropolitano de Lisboa.\\

Further, most automation projects at the international level attempt to associate location to a city level. While this level of granularity is likely sufficient for most international applications that seek to evaluate trends on a grande scale, it brings no further insight to hyperlocal exploration. Local officials interested in monitoring on a parish or even neighborhood (and therefore not administratively defined) level will gain no further insight from associations to the city as a whole. Likewise, as is clear from the previous discussion on placemaking and spatiality, users bring a variety of realities in association with places. Among the different types of users of the city (at the work vs. play vs. live level), the name of place will be understood differently. Moreover, neighbors within a particular parish may define the same neighorhood in different ways, or use different language to describe it. This process allows publishers with a clear understanding of where an event (as the subject of his or her article) is occuring (or was or will occur) to define its boundaries outside of common understandings or administrative definitions. This provides a common defintion of place that can be visually undertsood, and persist beyond changing borders or evolving names.\\


%-{\color{orange}“To lead industry change, trailblazers: 1. Tell their story of a better future for the industry. 2. Show the way with their innovative circular offer that sets a new standard. 3. Freelly share their insights to inspire informed followership. 4. Initiate collaboration to further develop and spread their type of solution. 5. Influence public policy to propel circular solutions”\cite{WEF2021}}\\
%-{\color{orange}“Trailblazers operate in local innovation ecosystems. Four elements of ecosystem support show promise to increase chances of trailblazers achieving systemic impact. 1. Platforms for storytelling that help them gain much-needed credibility. 2. Knowledgeable investors who provide them with patient capital. 3. Consultative policy-makers who enable trailblazers to accelerate the circular transition in an inclusive way. 4. A high degree of connectivity throughout value chains to find the right ambassadors, like-minded customers and innovation partners”\cite{WEF2021}}\\




%While other geolocation news proj-{\color{orange}“we will investigate human-in-the-loop methods to improve the geolocation prediction. For instance, it might be possible for users to explicitly provide city labels or non-geotagged tweet, or correct predicted city labels.”\cite{Snyder2019}}{\color{purple}this project expands upon this concept}\\




%-{\color{orange} “using social media and mobile technologies as tools to increase two-way interaction between citizens and (local) governments will not reduce the workload of professionals.”} \cite{Kleinhans2015} {\color{purple} additional workflow to manage information and incorporate ideas from citizens.} \cite{Kleinhans2015}\\




%\subsection{User experience}
%This project aims to elaborate a tool that is easy to navigate and understand without any prior knowledge of or experience with GIS\cite{Evans-Cowley2010, Brown2012}. .%-{\color{purple} Aim to create a tool that is easy to navigate and understand without prior GIS knowledge. Similar level as standard applications such as AirBnB, Uber, Glovo, etc.  Additional API and dashboards options for more advanced users.} \cite{Evans-Cowley2010} -{\color{purple} The tool must be easy enough for non GIS trained people to define place easily and efficiently with the given tools. Leverage well-utilized UIs and tools (google maps or similar) to ease the usability and increase precision\cite{Brown2012}}\\
%Many daily products have already situated the modern urbanite to use temporal, spatial, and thematic filtering by integrating dynamic online maps (such as AirBnB, Uber, Idealista, etc.). While this will be sufficient for the browsing, preference setting, and searching uses of the informal user, more advanced users should be able to leverage an API connection and monitoring dashboards to glean additional insight from the collected spatial data. Wherever possible, well-utilized and therefore comfortable user interfaces and tools will be implemented to ease the usability and increase precision.
%
%The tool should include a method to display relevant information in non-digital formats so that the same information can have an impact for non-tech savvy citizens or used in non-digital settings. %-{\color{purple}  The tools can be used offline to demonstrate to non-digital citizens the distribution of issues during planning meetings.} \cite{Evans-Cowley2010}\\
%
%Registration to view the site is not necessary and browsing is completely unblocked. Contributing data, however, requires the definition of an account, which will verify the user as belonging to a particular institution or self-defining as un-affiliated. Otherwise, no personal demographic data is required, as it can inhibit participation from the community\cite{Afzalan2017}. %-{\color{orange}“Asking participants to provide personal information (e.g. age, zip code, gender) as a part of their registration process may discourage their participation, but such inquiries can provide valuable information for planners in understanding the extent to which the online community represents community demographics.”}\cite{Afzalan2017} {\color{purple} The information on the tool is open and free.  Users need only create a profile if they wish to save searches or create dashboards, export info, etc.  What information should be requested from users to do so?}\cite{Afzalan2017}\\
%Should a user prefer to set a dashboard and experience hyperlocal siutation based on their saved preferences or current location, they may set and save such preferences \cite{Leszczynski2019}.%-{\color{purple}Value of geoloaction in platforms to user experience via hyperlocalization and centering platform services to the literal user \cite{Leszczynski2019}}\\
%
%The graphic user interface of each product should immediatley indicate to each user how they can interact with the tool, with the support of accompanying simple, explicit instructions. 
%A challenge of dashboards striking the appropriate balance between legible data presentation and valuable insights gained. Clustering techniques can be leveraged to ensure that key representations (such as density) are maintained without confusing the observer \cite{Teitler2008}.% -{\color{purple} Instead of the spread method of balanced marker modes, I prefer the clustering of ALL stories with a number indicating how many can be found in an area that is resolved by zooming in. \cite{Teitler2008}}\\ %To avoid geographically clustered visualization of uneven news coverage “Marker selection is therefore a tradeoff between story significance and spread. “To achieve a balance in marker mode, NewsStand divides the viewing window into a regular grid, and requires that each grid square contains no more than a maximum number of markers. The markers to display are selected in decreasing order of story significance and story age. This approach ensures a good spread of top stories across the entire map.
%A common user interface in both commercial and academic use is the integration of pop up bubbles that provide additional attributes about specific features when hovered or clicked upon, so that a viewer is neither overloaded with informaiton, nor lacking relevant context. \cite{Teitler2008}. %-{\color{orange}Hovering on a story indicator yields a bubble of story content and attributes\cite{Teitler2008}}\\
%The system should have filters that let researchers, decisionmakers, or browsers identify exactly the information they want to find. These include the type of story (articles, editorials, reports, etc.); tags; publishing affiliation; time; duration, location, etc. %  particular publishers, or to discern verified contributors (those associated with an institution and/or trained to use to tool as a part of their process) versus individual, volunteer based information. discerning verified contributors (those who have been trained to use/the tool is a part of their process), such as commercial or government organizations. % Individuals or smaller users can also contribute to the map, but may be filtered from these views as recreational or unaffiliated contributors(or similar) that can allow viewers/researchers/readers the ability to remove personal blog or other unverified content from cluttering the view (in addition to searching for specific sources, or checking/unchecking sources from a list to customize the contributors).\cite{Brown2012}}\\
%Previous literature also supports multiple views (Document, Candidate List, and Footprint) to provide different options for users to ingest the produced information \cite{Cai2016}, %-{\color{orange}Layout: documentView, CandidateListView, and FootprintView\cite{Cai2016}}\\
%as well as textual tags or symbols representing these to ease readability of the visual layout \cite{Karimzadeh2019}.%-{\color{red}Include textual tag in reference map, include different map symbols for each tag\cite{Karimzadeh2019}}\\
%An important element for decision makers is a platform that support geointelligence applications via additional processing and visualization techniques \cite{Imani2019}, which may be included in future iterations.%-{\color{purple}determining spatial patterns from news sources supports GEOINT and therefore the application and dashboard views for decision makers.\cite{Imani2019}}\\
%
%The tool should support only enough information per article to validate interest of the reader (the articles are not hosted here, just directed to the original source). The purpose of this tool is nuanced filtering and discover. 
%
%This tool is not intended to be an automatic extraction, yet rather a beginning of an industry pivot towards including spatial definition such that it does not need to be extracted in the future.  Those describing the content should have the ability to identify precisely the location (point or area) of interest at the point of publication (leaning on existing addresses, boundaries, or points of interest as appropriate, if custom defined areas aren’t necessary.





%-{\color{orange}“They build alliances with scale-up peers, multinationals, organizations within their value chain and throughout industries; and they collaborate with academia and non-profits. These collaborations help them combine resources for technology advancement, develop new industry standards and increase distribution and market access.”}\\


%\subsection{Application of results}
%%The project results will be licensed as free and open source such that these can be accessible and leveraged by other individuals or organizations for further development or related projects. Wherever possible, the project will leverage existing open source tools, platforms, and data. Thus far, all sources of data are already publically available, which mitigates concerns of distribution of proprietary materials.
%-{\color{purple} The tool can support the aggregation of public, private, and citizens knowledge to be used for more nuanced applications.
%Foundation for future development and sophistication of tools in an increasingly spatialized data context.} \cite{Afzalan2017}\\
%-{\color{purple}he tool is not an SDI but could contribute to a local SDI, like the one described in this paper, to support “geomatics for sustainable societies”.} \cite{Bhattacharya2018}\\
%-{\color{purple}Results should be sharable via GeoRSS. Format for distributing search results, easy search and linkin to other open projects.}\cite{Xing2015}\\
%-{\color{purple} Results should also be shareable via downloadable CSV, shapefile, etc.}\\
%-{\color{purple}Extract: value in research and monitoring to be able to save datasets (or better yet support integration in other platforms)\cite{Shneiderman1996}}\\

%\subsection{Structure}
%-{\color{purple}The tool should support only enough information per article to validate interest of the party (they can be read from the article sources themselves… this is not a blog hosting site), the rest should be left for nuanced filtering and discovery.\cite{Jiang2020}}\\
%-{\color{purple}Even in blog type scenarios… the tool should leverage existing blogsites to host individual content, adding only a plugin for a location element.  The tool does not host blog, only summaries and attribute data that links to the original source. The content is included within the spatial database only as an attribute such that it can be searched for keywords}\cite{Afzalan2017}\\
%-{\color{purple} This tool is not intended to be an automatic extraction, yet rather a beginning of an industry pivot towards including spatial definition such that it does not need to be extracted in the future.  Those describing the content should have the ability to identify precisely the location (point or area) of interest at the point of publication (leaning on existing addresses, boundaries, or points of interest as appropriate, if custom defined areas aren’t necessary).}\cite{Bhattacharya2018}\\
%-{\color{purple}The tool may leverage different coordination methods (top down and bottom up).  Historical data should be mapped via broker coordination, whereas anything included via the input portal (white labeled on no) can be considered “federated” coordination.  Focus on the broker to improve sustainability and open application of the tool. \cite{Jiang2020}}\\
%-{\color{purple}Expected use of 1D, 2D, Temporal, Multi-dimensional, tree (related articles) data types\cite{Shneiderman1996}}\\





%\subsection{Impact}
%\begin{enumerate}
%%	\item 1+ news publication organizations
%	\item 1 webapp, freely and openly accessible, available in English and Portuguese languages
%	\item 1 webapp development code, open licensed for further or related future development by any individual or organization.
%\end{enumerate}
%
%\subsection{Concerns}
%-{\color{purple}What kind of potential themes may emerge (accurately or otherwise) of places in Lisbon? How should these be dealt with? This project doesn’t have any visibility, yet it is meant to identify trends -- both actual (because such events are happening in  a particular pattern) and contrived (only events of a certain type are being covered in certain places by journalists).\cite{McQueenBaker2019}}\\

%{\color{red} 
%*Incorporate verbage from section 1.3 to 2 of proposal\\
%*Organize from here\\
%}
%

